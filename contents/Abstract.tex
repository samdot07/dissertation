\begin{abstract}
    This thesis explores the intricate relationship between capitalism and slavery in Britain from the seventeenth through the nineteenth centuries. It examines how the burgeoning economic principles of capitalism fueled and sustained the Atlantic slave trade, generating wealth crucial for funding Britain's industrial revolution and essential advancements. The thesis argues that slavery wasn't just a moral issue but a vital part of the capitalist system that drove Britain's economic growth during this period. Eric Williams' work, Capitalism and Slavery, serves as a foundation, highlighting how British capitalism not only allowed humans to be treated as commodities but also relied on slave labor to maintain economic progress. By analyzing economic policies, industrial development, and trade networks linked to the slave economy, this thesis reveals how profits from slavery impacted British society, from merchant elites to emerging industrialists. Moreover, the thesis examines the beliefs that justified slavery within British capitalism, exploring how idea of race, labor exploitation, and profit maximization became connected with imperial ambitions and global dominance. Every source and historical records analyzed through which Williams' study became reality has been properly cited, enabling this study to focus on how economic interests, political power, and the moral contradictions blended in a capitalist society built upon slavery. In conclusion, this thesis argues that the history of capitalism in Britain cannot be fully understood without recognizing the role played by slavery. By critically evaluating Eric Williams' perspective, it offers insights on how slavery shaped British capitalism and vice versa, leaving a significant legacy that continues continues to influence debates today about economic justice and reparations.
\end{abstract}