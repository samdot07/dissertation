THE DISCOVERY OF AMERICA and the maritime passage to India were two of the most momentous occurrences in history. America's discovery was significant because it created a large market for European products. This incident dramatically improved the mercantile system \endnote{\protect\fullcite[pp. 415–416, 590–591.]{smith}.}, resulting in large increases in worldwide trade. Trade dominated the seventeenth and eighteenth centuries, whereas the nineteenth century centered on industry. Much of Britain's trade was conducted through the triangle trading system. In 1718, William Wood identified the slave trade as the origin of other trades \endnote{\fullcite[pt. 3, p.193.]{wood}}, and Postlethwayt subsequently referred to it as the foundation of the whole system \endnote{\fullcite[p. 143.]{rees}}. In this commerce, England gave products and ships, Africa supplied slaves, and plantations provided raw resources. English ships took manufactured commodities to Africa, exchanged them for slaves, and then traded these slaves on plantations for raw produce, which was shipped back to England. This period boosted new industries in England. Even as trade volumes grew and direct commerce between England and the West Indies established, triangular trade remained important. The triangle trade significantly boosted British manufacturing. Maintaining slaves and their owners on plantations established a market for British products, as did New England agriculture and Newfoundland fishing. By 1750, practically every commercial or industrial town in England was involved in triangular or direct colonial commerce \endnote{\fullcite[p. 111.]{gee}}. Profits from this trade made a considerable contribution to capital accumulation in England, which fueled the Industrial Revolution. The West Indian islands became crucial to the British Empire, contributing significantly to England's riches and influence. The slaves made these sugar colonies extremely lucrative \endnote{\fullcite[pp. 4, 6.]{postlethwaytAT}}. Sugar plantations in the West Indies generated far larger earnings than any other type of cultivation in Europe or America \endnote{\fullcite[p. 366]{smith}.}, which thrilled mercantilists. The triangle trade and commerce with the sugar islands were more important to England than her tin or coal resources due to the shipping they promoted \endnote{\fullcite[pp. 50-54.]{bennet}}. Unlike the mainland, these tropical colonies did not compete with England's industries and shown no evidence of industrialization, which was a continual source of anxiety for the mainland. Their significant black population meant they didn't pursue independence \endnote{\fullcite[pp. 13-14.]{postlethwaytAT}}. This problem may be summed in a single word, sugar. According to Sir Dalby Thomas, sugar boosted the pleasure, glory, and grandeur of England more than any other item, including wool \endnote{\fullcite[p. 82.]{ellis}}. However, there was one fundamental issue, monopoly. The economic ideology of the period did not encourage free commerce, and colonial trade was strictly regulated by the home nation. Mercantilists argued that the colonies owed their wealth to the home nation and had a responsibility to support it \endnote{\fullcite[p. 104.]{see}}. The colonies were expected to send their valuable products only to England and aboard English ships. They could only buy British goods unless foreign items were first imported into England. This assured they worked for England's interest while remaining focused on agriculture. In return, England allowed colonial products a monopoly on the domestic market. The Navigation Laws were important to this system and were regarded as English measures designed for English ends \endnote{\fullcite[p. 242.]{harper}}. These regulations targeted the Scots, Irish \endnote{\fullcite[vol. 4, pp. 65, 71, 126, 154–155.]{andrews}}, and Dutch, who first helped British colonies by offering lower rates than the British \endnote{\fullcite[vol. 4, p. 9.]{andrews}}. The Navigation Laws recognized slaves from Africa and sugar from the West Indies as important commodities. Despite these constraints, West Indian sugar growers never fully accepted the limitations on their trade. In 1739, they obtained a minor amendment to the Navigation Laws, but it was restricted and only permitted commerce with poor foreign markets in Europe, offering little profit. Even this modest concession infuriated English businessmen \endnote{\fullcite[vol. 4, p. 828.]{stock}}. A century later, the fight between monopoly and free trade resurfaced. British businessmen and manufacturers, now campaigning for free trade, fought with West Indian sugar growers, who had shifted their position to preserve the monopoly they had earlier attacked, claiming it made them the merchants' slaves \endnote{\fullcite[vol. 2, p. 264]{andrews}}.
\subsection{shipping and shipbuilding}
This international trade inevitably resulted in a major increase in shipping and shipbuilding. One significant advantage of the triangle trade was that it helped to expand England's naval navy. Merchant ships were not much different from battleships back then. Shipbuilding in England benefited directly from the triangle trade. Ships of a special design were built for the slave trade, with an emphasis on both capacity and speed to lower fatality rates. The maritime industry, like other industries, differed in its approach to managing the slave traffic. Some backed the Royal African Company, while others preferred free trade \endnote{\fullcite[vol. 3, pp. 204n, 207n, 225n, 226, 249, 250n, 251.]{stock}}. However, on the topic of abolition, the industry stood unified, saying that abolishing the slave trade would damage Britain's naval and imperial supremacy. Aside from sailors, several other crafts relied on the shipping business, including carpenters, painters, boat builders, and numerous artists involved in repairs, equipment, and loading, all of which were dependent in some way on ships dealing with Africa \endnote{\fullcite{holtgregson}. \parencite{ericwilliams}.}. It was expected that almost everyone in town would experience the repercussions of abolition, either directly or indirectly. Sugar-producing islands also contributed to the expansion of shipping. The West Indies' economy was based on crop exports and food imports, with fish serving as a key product. Fish was an integral part of the nutrition of slaves on plantations, and it also helped England's herring trade, which found a large market in the sugar plantations \endnote{\fullcite[p. 337.]{macinnes}}. The rise in shipping put enormous strain on England's docks in the eighteenth century. Warehouse space on the quays was insufficient for the number of imports, and coal ships experienced delays, leading coal prices to skyrocket. This condition resulted in organized crime. To solve these issues, West Indian merchants established a separate constabulary to prevent theft and a register of laborers who unloaded West Indian ships. They also successfully pushed Parliament for the building of the West India Docks, which gave them a twenty-one year monopoly on loading and unloading boats participating in the West Indian trade \endnote{\fullcite[vol. 1, pp. 76–82, 89–108.]{broodbank}}.
\subsection{the seaport towns}
The expansion of triangle trade routes, as well as advances in shipping and shipbuilding, had a considerable impact on the establishment of major coastal towns. During the trading era, Bristol, Liverpool, and Glasgow developed as significant centers, much like Manchester, Birmingham, and Sheffield would later become essential during the industrial revolution. Bristol's rise to become England's second-largest city in the eighteenth century was largely due to its engagement in the slave and sugar markets. A local chronicler said, there is not a brick in the city but what is cemented with the blood of a slave \endnote{\fullcite[vol. 3, p. 165.]{nicholls}}. When Liverpool eclipsed Bristol in slave trade activity, Bristol shifted its emphasis from triangle trading channels to straight sugar commerce. This move resulted in a significant rise in the number of ships traveling directly to the Caribbean \endnote{\fullcite[p. 233.]{macinnes}}, where the West Indian trade produced double the value of all other international commerce combined \endnote{\fullcite[pp. 358,370.]{macinnes}}. Bristol had its own West Indian Society, and the Town Council supplied financial assistance to sugar islands damaged by fire. Younger members of West Indian corporations sometimes worked on plantations before coming home to start their own businesses. Throughout the eighteenth century, Bristol's members of Parliament were regularly linked to sugar plantations. The West Indian trade was so important that until the mid-nineteenth century, Bristol had a representation in Parliament representing West Indian interests. Bristol was passionately opposed to measures to equalize sugar duties, which ultimately contributed to the downfall of its West Indian commerce until the late nineteenth century, when the banana trade restored it \endnote{\fullcite[p. 371.]{macinnes}}. Liverpool's importance in the commercial world by 1770 owed primarily to its involvement in the slave trade \endnote{\fullcite[p. 109.]{mantoux}}. Abolitionists contended that Liverpool's rise was also influenced by the salt trade, Lancashire's fast population increase, and the expansion of Manchester's industries \endnote{\fullcite{clarkson}.}. However, Liverpool's amassed riches from the slave trade drew people to Lancashire and fueled Manchester's industrial expansion \endnote{\fullcite[vol. 2, p.690.]{corry}}. Abolitionists were concerned that abolishing the slave trade would damage Liverpool, which was deemed vital to its economy and national riches \endnote{\fullcite[vol. 1, p. 256.]{picton}}. Despite the controversy, Liverpool's tenacity and entrepreneurial drive are likely to have resulted in prosperity in other industries, although at a slower rate, in the absence of the slave trade \endnote{\fullcite[vol. 2, pp. 589, 745.]{touzeau}}. Glasgow's membership in colonial trade following the 1707 Act of Union with Scotland raised its standing tremendously. Sugar and tobacco contributed significantly to Glasgow's economic prosperity in the eighteenth century, spurring the development of new businesses. As Bishop Pococke observed after his visit in 1760, Glasgow has benefited tremendously from the Union, particularly through its trade with the West Indies in tobacco, indigo, and sugar \endnote{\fullcite[vol. 3, p. 295.]{eyretodd} \parencite{ericwilliams}.}.
\subsection{the goods}
We must now look at England's industrial growth, which was either directly or indirectly influenced by commodities involved in the triangle trade and the processing of colonial output. The creation of these commodities boosted capitalism, created jobs for British workers, and resulted in significant profits for England.
\subsubsection{wool}
Prior to the Industrial Revolution, wool dominated English industry, overshadowing others. It had an important part in the slave trade beginning in the late seventeenth century. Slave ships frequently carried woolen items as part of their cargo, and in 1695, a legislative commission recognized that trade with Africa helped the wool industry \endnote{\fullcite[vol. 2, pp. 109.]{stock}}. English wool producers were heavily involved in the divisive conflict between the Royal African Company and independent dealers. Proponents of the corporation said that uncontrolled traders disturbed the market and caused decreases when the company's monopoly was challenged. Clothiers typically backed the company's monopoly \endnote{\fullcite[vol. 2, pp. 162n, 186n.]{stock}. \fullcite[vol. 3, pp. 207n, 302n.]{stock}. \fullcite[vol. 1, pp. 413, 417-418.]{donnan}}, whereas wool interests favored free trade \endnote{\fullcite[vol. 1, pp. 379.]{donnan}}.According to evidence from a customs house official, unfettered commerce resulted in increased wool exports, but wool merchants complained in 1694 that restrictions significantly reduced their sales \endnote{\fullcite[vol. 2, pp. 29n, 89n, 94, 186n.]{stock}}. Other petitions underlined the importance of colonial markets for the wool industry. Despite the fact that lighter materials are preferred in tropical areas, England purposefully marketed woolen items there \endnote{\fullcite[vol. 3, p. 45.]{stock}}. Cotton eventually surpassed wool in both colonial and domestic markets. By 1783, the wool industry had begun to embrace technical innovations comparable to those that had changed the cotton sector previously. Following this time, the triangle trade and West Indian markets were less important to the wool business.
\subsubsection{cotton}
Manchester profited in the eighteenth century by making cotton items needed to purchase slaves, much as Liverpool did by building ships for the slave trade. Manchester, also known as Cottonopolis, grew at first due to demand from Africa and the West Indies. This expansion was inextricably linked to Liverpool, which offered access to worldwide markets and directed revenue from the slave trade into Manchester's development. Manchester's cotton goods were transported to Africa on Liverpool slave ships, highlighting the importance of the triangular trade in Manchester's growth. Indian textiles, which were prohibited in England, dominated the African market, further linking Manchester's interests to the slave trade. Despite efforts to compete with the East India Company, Manchester struggled to match the vibrant hues of Indian cottons. Manchester, on the other hand, became well-known for its cotton and linen checks. European and colonial warfare throughout the 1730s and 1740s, as well as changes within the African Company until 1750, hampered the cotton trade with Africa. After 1750, limited Indian exports pushed English manufacturers to seize the opportunity by pushing their own products. By 1780, checks had ceased to be an important element of Manchester's cotton business, owing in part to military interruption. Manchester could only fulfill African market demand when Indian cotton was in short supply or at exorbitant rates. The plantation market demanded affordability, but by 1780, raw cotton had become increasingly expensive due to rising demand and restricted supply expansion \endnote{\fullcite[pp. 147–166.]{wadsworth}} Manchester prospered from colonial trade by delivering commodities to slave coasts and plantations while relying on imported raw materials from the Levant and the West Indies in the eighteenth century. Despite severe competition from India, which threatened local markets, Manchester gained supremacy in England's domestic market by imposing exorbitant levies on Indian imports. This prompted private Indian businessmen to import raw cotton for Lancashire manufacturers. England initially relied on the West Indies for raw cotton in the early eighteenth century. However, with the invention of the cotton gin and better access to mainland cotton, Manchester's cotton exports rose dramatically during the American Revolution, moving into new markets in the newly formed United States \endnote{\fullcite[pp. 169.]{wadsworth}}. Cotton commerce evolved into a worldwide operation that extended beyond local markets. This transformation presented issues for the West Indies, whose expensive cotton failed to compete with mainland output. Political upheavals lay ahead, portending an uncertain future for the West Indies.
\subsubsection{sugar}
The extraction of raw resources from colonies fueled the growth of new industries in England, boosted employment in shipping, and improved international trading opportunities. Sugar, in particular, had a significant impact, giving rise to the sugar refining business. As sugar became a need rather than a luxury, plantation output increased, emphasizing the importance of refining. This business also boosted secondary trades and required transportation infrastructure for distribution \endnote{\fullcite[p. 302.]{latimer18}. \fullcite[p. 340.]{pitmanDev}}. Bristol grew as a major sugar refining hub, outnumbering London in refinery numbers relative to its size \endnote{\fullcite[p. 70.]{newbristolguide}}. In 1789, Bristol refiners opposed the prohibition of the slave trade, which was critical to the West Indies' economy \endnote{\fullcite[vol. 2, pp. 602-604.]{donnan}}. While Glasgow is frequently linked with tobacco, the city's wealth in the eighteenth century was largely due to sugar refining. Due to legislative constraints on direct commerce with colonies, Glasgow had to acquire its raw sugar from Bristol before to 1707. The 1707 Act of Union lifted this limitation, which benefited Glasgow's sugar refining industry \endnote{\fullcite[vol. 3, pp. 39–40, 150–154.]{eyretodd} \parencite{ericwilliams}.}. Most refineries were located in and around London. Why wasn't raw sugar refined on the plantations? Originally, England outlawed refining in colonies to protect its economic interests. This approach increased England's commerce and earnings \endnote{\fullcite[vol. 1, pp. 385, 390.]{stock}} but caused tension with sugar growers, who controlled the local market and kept prices high. Sugar imports fell during the American Revolution, resulting in higher costs. Distressed refiners petitioned Parliament for aid, citing their struggle with planters. Planters gained from high prices, while refiners lobbied for more sugar supply at reduced prices. These conflicts continued, driven by mercantilist rules governing sugar production and trading.
\subsubsection{rum}
Another colonial resource fueled the growth of a new English industry. Molasses, a large byproduct of sugar production utilized in rum distillation, played an important role. However, unlike cotton or sugar, rum did not have the same economic impact on British industry. This was largely owing to the significant importation of finished rum from the Caribbean islands. Rum had important roles in fisheries, the fur trade, naval provisioning, and the triangular trade, notably aboard slave ships \endnote{\fullcite[p. 285.]{saugnier}}. Distilleries enabled the manufacturing of molasses-based rum and other spirits, illustrating Bristol's close relationship with sugar plantations. Distillers often petitioned Parliament to protect their interests and against the introduction of French brandies. West Indian sugar planters claimed that rum accounted for a sizable amount of their overall exports, warning that its ban would destroy their livelihoods and push buyers to foreign replacements. They highlighted that the issue was not whether individuals drank, but rather what they drank \endnote{\fullcite[pp. 8-9.]{ginrum}}. Critics argued that the West Indian rum trade had a greater influence on British society than its economic gains \endnote{\fullcite[vol. 4, p. 310.]{stock}}. Molasses also caused tensions between West Indian sugar planters and English landowners, as well as with mainland colonists. West Indian interests regularly urged for the use of molasses as a replacement during grain shortages, but mostly to control sugar surpluses. The major challenge to West Indian distillers, however, came from their New England rivals rather than English farmers. New England traders preferred molasses over West Indian rum, distilling it for transport to Newfoundland, Native American tribes, and Africa, where it dominated the slave coast rum trade \endnote{\fullcite[vol. 1, p. 118.]{emory}}. This resulted in increased commercial interest in the triangular trade. However, the availability of cheaper French West Indian molasses presented a challenge to British interests. Instead of utilizing it at home, West Indian planters sold molasses to mainland colonists, prompting them to further rely on French supplies. This move has far-reaching ramifications for British sugar growers throughout time.
\subsubsection{pacotille}
The slave cargoes were deemed completed without the provision of pacotille, a term used to describe different articles and trinkets that appealed to Africans' penchant for vivid colors. Interestingly, even after engaging in internal commerce, Africans in the late nineteenth century would swap these items for land and mining concessions. The demand for glass products and beads remained constantly strong along the slave coast, while estates required vast amounts of bottles, the majority of which were made in Bristol \endnote{\fullcite[vol. 2, pp. 307-308.]{corryevans}. \fullcite[pp. 296–299.]{saugnier}.}. Although each item had little worth on its own, they formed a major transaction when taken together.
\subsubsection{metals}
Slave commerce required things that were equally important as wool and cotton supplies, but more melancholy. Fetters, shackles, and padlocks were essential for safely confining African slaves on ships and preventing revolt and suicide. Slaves were marked with branding irons. Iron bars, which were roughly equivalent to four copper bars, were used as payment over most of the African coast \endnote{\fullcite[vol. 1, pp. 234n, 300n.]{donnan}}. Both the Royal African Company and iron producers discovered lucrative markets in Africa. Guns were a standard component of any goods shipped to Africa. Birmingham was well-known for its gun manufacture, much like Manchester was for cotton. Birmingham weapons were sold for slaves in the eighteenth century, and it was widely assumed that the cost of one slave was equivalent to that of a Birmingham gun. Birmingham also provided muskets to Africa as a key customer, alongside the British government and the East India Company \endnote{\fullcite[pp. 145-146.]{court}}. Plantations also needed products \endnote{\fullcite[p. 195.]{ashton}}. Birmingham sent wrought iron and nails to plantations, although the commerce varied with the sugar price. Along with iron, exports included brass, copper, and lead. Brass pots and kettles were first sent to Africa around 1660, and they rose dramatically after 1698 due to open commerce. Throughout the eighteenth century, Birmingham exported large amounts of cutlery and brassware, successfully competing with foreign and colonial markets \endnote{\fullcite[pp. 137–138, 149–151, 286–292.]{hamilton}}. Shipbuilding boosted heavy industries. Many iron chain and anchor foundries in Liverpool flourished on shipbuilding. The need for copper sheathing for ships fueled the expansion of local industry in Liverpool and the surrounding areas \endnote{\fullcite[pp. 117, 119, 126–127, 130.]{ronald}}. Iron masters were interested in the slave trade throughout the century. When Parliament debated abolition, Liverpool producers of iron, copper, brass, and lead opposed it, anticipating unemployment and an outflow of people seeking work elsewhere \endnote{\fullcite[vol. 2, pp. 610-611.]{donnan}.}. Birmingham also acknowledged its heavy reliance on the slave trade for its many businesses. Concerns about abolition \endnote{\fullcite[vol. 2, pp. 609.]{donnan}.} were frequently overblown. The need for armaments during the eighteenth century commercial battles had primed iron masters for the far higher demands of the Revolutionary and Napoleonic battles. Despite technical advances, colonial markets failed to absorb greater output. Between 1710 and 1735, iron shipments nearly quadrupled \endnote{\fullcite[344–346, 347–355.]{scrivenor}}, indicating domestic economic expansion and shrinkage on the sugar islands. By 1783, iron masters began to reassess their position. However, the economic success was too attractive to end.