UNLIKE OUR TOBACCO colonies, which did not generate affluent planters like those from our sugar islands \endnote{\fullcite[p. 158]{smith}.}, sugar planters were among the wealthiest entrepreneurs throughout the mercantilist era. During the eighteenth century, these West Indian planters were well-known in English society. Their existence was owing to absentee landlordism, a common problem in the Caribbean, where the weather was pleasant. After making their wealth, many slave owners returned to Britain. According to historical records, England drew everything that was precious from the colonies like a magnet. It served as the focal point for all activities. Nothing but England can we relish or fancy. Our hearts are here, wherever our bodies be. All that we can rap and rend is brought to England \endnote{\fullcite[p. 113.]{pitman} \parencite{ericwilliams}.}. When these colonists returned to England, they hoped to buy estates, merge with the nobility, and conceal their roots. However, their influence frequently had a deleterious impact on English culture and morality \endnote{\fullcite[p. 125.]{pitman} \parencite{ericwilliams}.}. Their vast riches enabled excessive spending, which occasionally contrasted with the relatively frugal English nobility. Absenteeism had a tremendous impact throughout the Caribbean. Overseers and attorneys failed to manage plantations effectively. Governors occasionally struggled to assemble councils owing to absence. Furthermore, the unequal ratio of white to black people grew, increasing the likelihood of slave rebellions. Legislative attempts to reduce absenteeism were unsuccessful because absentee planters opposed plans to disperse idle lands into smaller farms \endnote{\fullcite[pp. 8-20.]{ragatz}. \fullcite[pp. 117-121.]{pitman} \parencite{ericwilliams}.}. West Indian riches became proverbial. Affluent West Indian communities in London, Bristol, and Southampton preserved their once-prestigious social position \endnote{\fullcite[p. 51]{ragatzFP}}. Many West Indian youngsters went to public schools in England \endnote{\fullcite[Duke of Clarence, July 5, 1799. vol. 34, 1102.]{parlhist}. \parencite{ericwilliams}.}. When planters' carriages collected in London's streets, they frequently created traffic jams. In several occasions, ordinary people in England rose to prominence through unexpected inheritances from West Indian plantations, which were highly valued in the eighteenth century. West Indian merchants also prospered, making significant profits from trade and influencing British business. By 1761, almost every important British merchant had a commercial tie to the West Indies \endnote{\fullcite[vol. 1, p. 210.]{namier}}. Planters and merchants initially disagreed on price, but by 1780, they had formed an alliance to protect their trading interests against the expanding free trade movement \endnote{\fullcite[pp. 185-187.]{penson}}. They created the West India interest, a powerful organization that included colonial agents from England. In the corrupt politics of the day, their riches allowed them to influence elections, buy votes, and obtain seats in Parliament. To maintain their power, West Indians, like slave dealers, were entrenched not just in the House of Commons but also in the House of Lords. Moving between these houses was customary, and peerages were frequently traded for political influence. According to some estimates, nearly every aristocratic family in England has connections to the West Indies \endnote{\fullcite[p. 80.]{thierry} \parencite{ericwilliams}.}. They had power outside Parliament by serving as mayors, councilors, and alderman. Initially supported by powerful friends, the West Indian interest soon lost popularity owing to their refusal to compromise on slavery. Together with other powerful monopolies—the landed nobility and wealthy merchants from port cities—they exerted enough influence in Parliament to give any politician pause \endnote{\fullcite[pp. 59-60.]{prinsep}}. They passionately fought abolition, emancipation, and any changes to their monopoly. They persistently fought increased sugar tariffs. Until American Independence provided a substantial blow to mercantilism and monopolies, the West India interest continued to be a disruptive factor in English politics. In 1744, planters urged every member of Parliament to oppose planned sugar duty hikes. They effectively transferred the additional taxes to foreign linens. This episode demonstrated the difficulties of enforcing increased sugar tariffs owing to powerful interests \endnote{\fullcite[February 20, 1744. vol. 13, pp. 652, 655.]{parlhist}. \parencite{ericwilliams}.}. The heyday of the West India sugar interest faded with the turn of the century and the reformation of Parliament, resulting in the emergence of the Lancashire cotton interest. Unlike their predecessors, this new generation preferred laissez-faire to monopolies.