BRITAIN BENEFITED tremendously from the triangular trade. The increased demand for consumer products resulting from this trade undoubtedly boosted the country's industrial capacity. This industrial expansion necessitated financial investment. Few people in the early to mid-1700s had the financial resources of West Indian sugar growers or Liverpool slave traffickers. We have already discussed how absentee plantation owners purchased land in England to support important advancements during the Agricultural Revolution. Now, let's look at how revenues from the triangle trade were invested in British industry. These funds were critical in financing the development of huge plants to satisfy the needs of new manufacturing technologies and growing markets.
\subsection{investments and profits}
\subsubsection{banks}
During the eighteenth century, banks in Liverpool and Manchester were critical to the triangular commerce. Liverpool, notorious for its slave trade, and Manchester, a cotton industry powerhouse, relied on these banks to support cotton mills and canal improvements that improved communication between the towns. Bankers in that era often started as traders, advanced to merchants, and finally became bankers themselves. For example, the Heywood Bank, established in Liverpool in 1773, operated privately until it was acquired by the Bank of Liverpool in 1883. Its founders, prominent merchants who eventually sat in the Chamber of Commerce, were actively involved in commerce with Africa from 1752 until the slave trade was abolished in 1807 \endnote{\fullcite[pp. 91–97, 101.]{hughes}. \fullcite[vol. 2, pp. 493, 656.]{donnan}}. Thomas Leyland entered the financial scene in the early 1800s, although his participation in the African slave trade dates back to the late 1700s. He was one of Liverpool's most renowned slave merchants, acquiring considerable money. In 1802, Leyland rose to the position of senior partner in the banking business Clarkes and Roscoe. His company with fellow slave dealer Bullins, known as Leyland and Bullins, lasted ninety-four years till the bank merged in 1901 \endnote{\fullcite[pp. 170–174.]{hughes}. \fullcite[vol. 2, pp. 646–649.]{donnan}}. The Heywoods and Leylands' stories are significant, but they also mirror the greater financial scene in eighteenth-century Liverpool. Other notable bankers, including as William Gregson, Francis Ingram, and Captain Richard Hanly, were also involved in the slave trade, frequently interacting with key organizations like the Liverpool Fireside, which included ship captains, slavers, and merchants \endnote{\fullcite[pp. 74–79, 84–85, 107–108, 111, 133, 138–141, 162, 165–166, 196–198, 220– 221.]{hughes}. \fullcite[vol. 2, pp. 642–649.]{donnan}}. Similar dynamics emerged in Bristol, London, and Glasgow. Bankers in Bristol, such as William Miles, were not only involved in the banking industry but also vehemently opposed the abolition of slavery \endnote{\fullcite[pp. 297–298, 392, 468, 507.]{latimer18}. \textit{Annals of Bristol in the Nineteenth Century}, 113, 494}. The Barclay family was prominent in London's financial history, having amassed enormous riches through trading in America and the West Indies as well as banking \endnote{\fullcite[vol. 3, 235, 242– 243, 246–247, 249.]{barclay}}. The emergence of the Ship Bank in 1750 marked the beginning of Glasgow's banking industry, which was sponsored by people such as Andrew Buchanan and Alexander Houston, both of whom were important in the West Indian trade. Despite initial triumphs, initiatives like Houston's had disappointments, such as the attempted abolition of the slave trade in 1795, which resulted in significant financial losses. Overall, banking in these towns during the eighteenth century was inextricably linked to foreign commerce, including slavery, which significantly shaped their economic landscapes.
\subsubsection{heavy industry}
Heavy industry had an important role in propelling the Industrial Revolution and promoting triangular commerce. Capital from the triangular trade immediately aided the expansion of metallurgical industry. Notably, cash raised from West Indian commerce sponsored James Watt's pioneering work on the steam engine. Boulton and Watt were financially supported by Lowe, Vere, Williams, and Jennings, subsequently known as the Williams Deacons Bank. During the American Revolution in 1778, James Watt had difficulties as the French threatened the West Indian fleet. In a letter to Watt, Boulton offered optimism. Even in this situation, Lowe, Vere, and Company might yet be salvaged if the West Indian fleet comes safely from the French... as many of their investments rest on it \endnote{\fullcite[p. 113.]{lord}}. The bank survived, preserving Watt's ground-breaking idea. Sugar growers were among the first to acknowledge the engine's value. In 1783, Boulton wrote to Watt about supporters such as Mr. Pennant, Mr. Gale, and Mr. Beeston Long, major Jamaican sugar plantation owners who recognized steam power as a viable alternative to horses \endnote{\fullcite[p. 192.]{lord}}. Antony Bacon, a prominent ironmonger in the eighteenth century, was closely connected to the triangle trade through his partner, West Indian planter Gilbert Francklyn. Bacon ventured into the African trade, initially supplying troops with food before eventually offering skilled African laborers for government contracts in the West Indies. He founded an ironworks at Merthyr Tydfil in 1765, which expanded fast thanks to contracts during the American War, and another furnace in Cyfartha in 1776. Bacon imported iron ore from Whitehaven and helped enhance the harbor as early as 1740. His lucrative contracts to provide artillery to the British government greatly wealthy him \endnote{\fullcite[pp. 25–27, 32, 39, 41, 43.]{namierA}. \parencite{ericwilliams}. \fullcite[pp. 187-188.]{clapham}}. Beginning in 1753, William Beckford established himself as a notable ironmonger \endnote{\fullcite[vol. 2, p. 131.]{beaven}}. Part of the funding for the Thorncliffe ironworks, which began in 1792, came from Henry Longden, a razor manufacturer who inherited riches from his uncle, a successful West Indian businessman in Sheffield \endnote{\fullcite[p. 157.]{ashton}}.
\subsubsection{insurance}
During the eighteenth century, when the slave trade was extremely profitable and West Indian plantations were highly prized by the British Empire, the triangular trade was critical for new insurance businesses. Lloyd's began as a coffee shop and was regularly featured in advertising in the London Gazette about escaped slaves \endnote{\fullcite[p. 62.]{martin}}. The first known advertising addressing Lloyd's, from 1692, featured the auction of three ships bound for Barbados and Virginia. During the speculative bubble of 1720, the only enterprise mentioned by Lloyd's was commerce with Barbary, North Africa, and Africa. According to fire insurance expert Relton, Lloyd's insured against fires in the West Indies from a very early date. Lloyd's, like other insurers, insures slaves and slave ships and strictly adheres to judicial rulings defining natural death and perils of the sea. They backed heroic personalities such as a Liverpool captain who in 1804 resisted a French vessel while bringing commodities from Africa to British Guiana. One of Lloyd's prominent chairman, West Indian planter Joseph Marryat, successfully maintained Lloyd's monopoly on maritime insurance against a competing business in the House of Commons in 1810 \endnote{\fullcite[pp. 19, 91, 151, 212, 218–219, 243, 293, 327.]{wright}}. In 1782, the prominent West Indian sugar business played a key part in creating another insurance firm, the Phoenix, which was one of the first to open offices overseas—in the West Indies \endnote{\fullcite[.]{clapham}}. John Gladstone, a famous West Indian businessman, led the Liverpool Underwriters' Association, which was established in 1802 \endnote{\fullcite[pp. 240-241.]{wright}}.
\subsection{industry development}
Abbé Raynal, a famous thinker of his time with strong links to the French bourgeoisie, believed that individual work in the West Indies played an important part in world dynamics \endnote{\fullcite[pp. 78-79.]{callender}}. The triangular trade fueled Britain's industrial growth, increasing the country's total output. Lord Penrhyn, owner of Jamaican sugar estates and leader of the West India Committee, transformed the Welsh slate business with advances developed on his Carnarvonshire estate \endnote{\fullcite[pp. 156–157., 37, 91, 125, 204–208, 219.]{dodd}. \parencite{ericwilliams}. \fullcite[p. 32.]{fay}}. Similarly, Joseph Sandars, a prominent role in the Liverpool-Manchester railway project, resigned from the Liverpool Anti-Slavery Society in 1824, indicating a reluctance to face the sugar growers \endnote{\fullcite[Huskisson to Sandars, January 22, 1824. Add. MSS. 38745, ff. 182–183]{huskisson}. \parencite{ericwilliams}. \fullcite[vol. 1, p.93.]{francis}}. Other triangular trade personalities, such as General Gascoyne, John Gladstone, and John Moss, played important roles throughout this time period \endnote{\fullcite[April 25, 1806. vol. 5, 919.]{hansard}. \parencite{ericwilliams}. \fullcite[vol. 1, p. 123.]{francis}. \fullcite[pp. 323–324, 337.]{frederick}. \fullcite[pp. 197-198.]{hughes}.}. While the Bristol West India interest contributed greatly to initiatives such as the Great Western Railway \endnote{\fullcite[pp. 34-38.]{sommerfield} \parencite{ericwilliams}. \fullcite[pp. 189-190.]{latimer19}}, it is wrong to attribute all economic success to triangular commerce. The expansion of England's domestic market and the reinvestment of industrial earnings as critical elements. This industrial boom, which was first fueled by mercantilism, finally outpaced and altered it. By 1783, the steam engine had revolutionized coal mining and the iron industry \endnote{\fullcite[p. 166.]{lord}. \fullcite[pp. 86-87.]{scrivenor}.}. Iron bridges and railroads, as well as innovations made by people like Wilkinson, transformed industrial landscapes \endnote{\fullcite[vol. 2, p. 514n.]{jackman}}. Cotton, a key industry throughout the Industrial Revolution, prospered because to technologies such as the spinning jenny and the water frame \endnote{\fullcite[pp. 148, 170.]{wheeler}}. From 1783 until 1850, both heavy industry and cotton prevailed, upsetting long-standing monopolistic regimes that slowed growth. The revolution emphasizes the larger economic impact, namely England's general economic development. In the late eighteenth century, leaders such as Adam Smith and Thomas Jefferson noticed and discussed these developments. Smith challenged colonial monopolies, which he claimed hampered British output, claiming that actual economic progress happened despite these limitations. His key work, The Wealth of Nations, encouraged intellectual and revolutionary movements while criticizing mercantilist policies. In essence, his research highlights an era of substantial technical and economic development, indicating a shift away from old mercantilist practices \endnote{\fullcite[p. 577]{smith}.}.