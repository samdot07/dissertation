PROFESSOR PITMAN asserts that in 1770, the Middle Colonies and New England's prosperity depended heavily on the income generated by commerce with the West Indies \endnote{\fullcite[preface, p. vii.]{pitmanDev}.}. In the eighteenth century, sugar production ruled the West Indian sugar trade, with the islands providing Europe's primary supply of sugar, while the British mainland possessions took a backseat. Cromwell was overjoyed to have acquired Jamaica, highlighting its strategic value, and Barbados was seen as a priceless gem in the British crown \endnote{\fullcite[Governor Willoughby. Colonial Seiries, vol.5, pp.382.]{calendarstatepaper}. \fullcite[John Reid to Secretary Arlington. Colonial Seiries, vol. 5, p. 414.]{calendarstatepaper}.}. Because the northern colonies were predominantly agrarian and lacked large-scale plantations like those in the West Indies, which implied unavoidable economic competition with England, mercantilists viewed them with suspicion. Many sailors received their training in New England's thriving fishing industry. They were able to undercut English pricing for agricultural items in island marketplaces thanks to their location advantage. The financial stability of England was threatened by this. Petty said bluntly that New England was not in England's best interests \endnote{\fullcite[vol. 1, p. 564.]{cambridge}}. Relocating New Englanders to the Bahamas, Trinidad, Maryland, and Virginia was attempted more than once. Cromwell thought New England was unappealing and unproductive \endnote{\fullcite[vol. 1, pp. 497–499.]{andrews}}. The Council of State attempted to persuade residents of New England to relocate to Jamaica in 1655, but these radical ideas were not well received \endnote{\fullcite[Colonial Seiries, vol. 1, pp.429-430.]{calendarstatepaper}. \fullcite[vol. 2, p. 248.]{winthrop}.}. In contrast to selling agricultural products and meat, England would not have been able to export its more value manufactured goods if it had lost the provisions trade with the northern colonies. The northern colonies provided much of the food for the West Indian colonies \endnote{\fullcite[vol. 2, pp. 9, 21, 22.]{whitworth}}, which were mainly focused on sugar production. They were unable to shift resources to crops for food and animals. As a result, sugar took center stage throughout the West Indies \endnote{\fullcite[William, Beckford, vol. 5, p. 259.]{stock}}. The West Indian sugar monopoly was mostly supported by the colonies on the mainland \endnote{\fullcite[pp. 78.]{callender}}. Although mercantilism was subsequently found to be an inherently defective system, England willingly gave up this advantage. As a result, the colonies in North America were brought into the imperial economy and supplied the products that the slave owners and sugar plantations required. People from New England were thought to as America's Dutch. Later, the American South's cotton and rice sectors were sustained by the mixed agricultural of the northern and center colonies, which complimented the West Indies' specialized agriculture. Virginia and Barbados were receiving food supplies from New England as early as 1650 \endnote{\fullcite[p. 43.]{bidwell}}. Part of the reason for this trade agreement was mercantilism. The planters from the West Indies recognized the value of supplies from the mainland. Statesmen and colonial planters in Britain had this as a deliberate policy. Numerous items that were sent to the islands from New England might have been made nearby. If our island could feed itself, what would happen to New England trade? \endnote{\fullcite[John Style to Secretary Morrice. Colonial Seiries, vol. 7, p. 4]{calendarstatepaper}.} posed the question to a Jamaican farmer. In the absence of the sugar islands, the colonies on the mainland would have suffered greatly \endnote{\fullcite[Narrative and Disposition of Capt. Breedon concerning New England, October 17, 1678. vol. 10, p. 297.]{calendarstatepaper}}.Personal relationships reinforced economic links. On the mainland, West Indians possessed property, and vice versa. In North America, West Indian heiresses were in high demand \endnote{\fullcite[pp. 61, 63.]{whitson}}. The two regions were dependent on one another. In order to maintain harmony, the mainland needed to provide enough basics for the islands, and the islands needed to generate enough sugar and molasses for the mainland. The British sugar planters were against growing their crops or establishing new colonies because they thought this would distort the market and drive down prices. Because of its exhausted soil, Barbados started to decline early \endnote{\fullcite[Governor Willoughby. Colonial Seiries, vol. 5, p.167.]{calendarstatepaper}}. French colonies challenged British control with their lush soil and cheaper production costs \endnote{\fullcite[pp. 70-71.]{pitmanDev}. \fullcite[vol. 4, p. 97.]{stock}. \fullcite[p. 180.]{pares}}. The dread of the British was losing to their competitors in France. They were angry that their servants had left for other islands and resisted British colonies in Suriname \endnote{\fullcite[Renatus Enys to Secretary Bennet. Colonial Seiries, vol. 5, p.167.]{calendarstatepaper}}. Less valued colonies, like Canada, were kept by the British over more desirable ones, like sugar-producing Guadeloupe and Cuba \endnote{\fullcite[pt . 2, pp. 85–86.]{whitworthST}}. The West Indian interest was favored by the 1763 treaty. Despite their potential for economic growth, Cuba and Guadeloupe were restored to Spain and France, respectively. The commercial system and the sugar colonies declined after American freedom. Britain no longer placed much attention on the West Indies. Although American commerce with the West Indies was prohibited by the Navigation Acts, trading continued. Smuggling and tensions resulted from this. The Molasses Act of 1733 was difficult to implement \endnote{\fullcite[p. 39.]{taussig} \parencite{ericwilliams}.}, although it benefited sugar growers \endnote{\fullcite[p. 272.]{pitmanDev}}. It fueled more colonial unrest and aided in the American Revolution. In an attempt to stop smuggling, the Sugar Duties Act of 1764 instead increased colonial instability \endnote{\fullcite[p. 133.]{callender}}. British economic dominance was challenged by American independence. The West Indian planters suffered as well since they had to change to fit new markets \endnote{\fullcite[p. 86.]{whitson}}. After 1783, the French sugar colonies surpassed the British colonies in terms of prosperity. The British West Indies began to deteriorate at this point \endnote{\fullcite[G.D.8/349. West Indian Islands. Papers relating to Jamaica, 1783–1804 and St. Domingo, 1788–1800.]{chathampapers}. \parencite{ericwilliams}. \fullcite[p. 303.]{hilliard}.}. Following American freedom, British strategy began to turn away from the Caribbean and toward India. American imports benefited from the establishment of free commerce between Britain and the United States in 1783 \endnote{\fullcite[p. 230.]{merivale}}. The mercantile worldview was further undermined by Adam Smith's theories, which led to a change in emphasis towards profit and loss within the British Empire. In general, the British West Indian sugar colonies declined and the mercantile system came to an end with American freedom and the growth of free commerce.