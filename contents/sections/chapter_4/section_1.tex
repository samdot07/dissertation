PERCEIVING AMERICAN INDEPENDENCE as a tragedy in England and across the globe at the time, but it really signified the end of one era and the beginning of a new one. In this new period, the West Indian monopoly is no longer applicable. We must investigate how England's productive forces expanded, aided by the colonial system, and how that system finally became a barrier that needed to be removed. In June 1783, Prime Minister Lord North complimented the Quaker opponents of the slave trade for their compassion, but regretted that ending the trade was difficult because it was necessary for practically every European nation \endnote{\fullcite[June 17, 1783. vol. 13, pp. 1026–1027.]{parlhist}. \parencite{ericwilliams}.}. Slave traffickers and sugar growers were enthusiastic. The West Indian colonies remained highly valued, being the empire's most precious property. Those that noticed were able to see unmistakable signals of coming change. In the same year as the Battle of Yorktown, James Watt's second invention for rotary motion transformed the steam engine into a potent industrial instrument \endnote{\fullcite[p. 340.]{mantoux}}, kicking off an era of tremendous industrialization in England. While Admiral Rodney's triumph over the French protected the sugar colonies, Watt's advances in steam power were revolutionising industry. Concurrently, Henry Cort's invention of the puddling method transformed iron making. These advances paved the way for the massive rise of British capitalism, which resulted in important governmental reforms in 1832 and targeted monopolies, particularly those in the West Indies. By 1833, no British industry had entirely modernized, ancient organizational structures remained but were increasingly outmoded. Water power was used in the early stages of the Industrial Revolution, but steam power progressively replaced it. Steam power was still uncommon in industry at the turn of the nineteenth century. There was more difference between a spinning mill and a domestic workshop as they existed side by side between 1780 and 1800, than between a factory of that date and a modern one \endnote{\fullcite[p. 257.]{mantoux}}. The cotton industry was the dominant capitalist industry. The size of cotton mills was unsurpassed in British history. The first steam-powered spinning mill was established in England in 1785, and the first in Manchester in 1789. The first steam-powered loom mill was constructed in Manchester in 1806 \endnote{\fullcite[p. 22.]{redford}}. The cotton business quickly generated money and gained tremendous influence. The Manchester capitalist from his mountain, like Moses on Pisgah, beheld the promised land \endnote{\fullcite[p. 102.]{ericwilliams}}. Eli Whitney's innovation in the New World benefited the Old World by enabling British West Indian planters who were unable to fulfill Manchester's needs \endnote{\fullcite[p. 36n.]{buck}}. The West Indies, which had backed Manchester, lost importance when Manchester's industrialists acquired influence and sent their first representatives to Westminster. Meanwhile, tremendous progress was achieved in the metallurgical sectors, which were critical for the machines age. Britain after Waterloo clanged with iron like a smithy \endnote{\fullcite[vol. 2, p. 223.]{cambridge}. \fullcite{claphamIn}.}. The introduction of the hot blast in 1829 significantly lowered the amount of coal required for iron smelting \endnote{\fullcite[p. 297.]{scrivenor}}. Iron began to be employed for a variety of new applications, particularly in equipment production. Manufacturers built early textile equipment out of wood, but by the 1820s, specialist machine builders arose, ushering in the era of replaceable components. By 1832, iron masters had equaled cotton spinners \endnote{\fullcite[p. 189.]{claphamIn}} in terms of money and entrepreneurship. Both the cotton and iron industries were prepared to leave monopolies during the Reformed Parliament. In 1825, Britain removed its restriction on mechanical exports, a decision that had far-reaching implications. British rails became popular in France and the United States \endnote{\fullcite[p. 421.]{scrivenor}}. The once-dominant sugar farmers struggled to keep up. In my humble opinion the woollen cannot be too closely following the steps of the cotton trade \endnote{\fullcite[p. 276.]{mantoux}}. However, the woolen sector was sluggish to change, with archaic processes lasting longer. Britain's population increased dramatically, particularly in cotton concentrations. Until 1815, Britain relied on Spain, Portugal, and Germany for its wool. Captain John Macarthur imported merino sheep to New South Wales, which resulted in the first export of Australian wool to England in 1806. By 1828, Australian wool was highly prized, and it was expected that Britain would soon import the majority of its fine wool from Australia \endnote{\fullcite[vol. 2, p. 231.]{cambridge}}, which proved correct. Historically, Australia had a near-monopoly on wool, analogous to Mexico's on precious metals \endnote{\fullcite[p. 121.]{merivale}}. In the 1840s, the British Empire's focus switched from islands to continents, from tropical to temperate climes, and from plantations to towns. Britain's industrial might made her the dominating force in the globe. She clothed the globe, exported men and machinery, and became the world's banker. With the exception of India and Singapore—the key to the China trade—acquired in 1819, the British Empire was a geographical expression \endnote{\fullcite[p. 104.]{ericwilliams}}. British capital and production started thinking worldwide. Between 1815 and 1830, at least fifty million pounds had been invested more or less permanently in the securities of the most stable European governments, more than twenty million had been invested in one form or another in Latin America, and five or six million had very quietly found their way to the United States \endnote{\fullcite[p. 64.]{jenks}}. However, investment in West Indian plantations was negligible \endnote{\fullcite[Lord Redesdale, April 19, 1825. New Series, vol. 15, 385.]{hansardNS}. \parencite{ericwilliams}.}. British demand for Southern cotton spurred the expansion of the Cotton Kingdom in the United States. Southern banks sought financing in London \endnote{\fullcite[p. 75-76.]{jenks}}. Revolutions in Latin America created new trading prospects for Britain. Liverpool changed its attention from the West Indies to Latin America. Exports to the British West Indies have fallen \endnote{\fullcite[vol. 2.]{cambridgeFP}}, making them less essential to British capitalism. Judged by the standards of economic imperialism, the British West India colonies, a considerable success about 1750, were a failure eighty years later \endnote{\fullcite[p. 52.]{burn}}. The Navigation Laws were changed in 1825, allowing colonies to trade directly with any area of the world. This undermined the monopoly. The same year, sugar from Mauritius was admitted into Britain on the same terms as West Indian sugar. The colonial monopoly on the domestic market was critical for West Indian planters, but for British businessmen, specific legislation was unneeded. The bigger slave populations of the United States and Brazil afforded more lucrative markets than the West Indies.  Manchester manufacturers questioned the monopoly's worth \endnote{\fullcite[Milner Gibson, February 24, 1845. Third series, vol. 77, p. 1062.]{hansardTS}. \parencite{ericwilliams}.}, describing it as antiquated and onerous \endnote{\fullcite[p. 203.]{merivale}}. The West Indies, which were economically undeveloped in 1832, were viewed as an anachronism \endnote{\fullcite[p. 73.]{burn}}. Mercantilism was finished, and political adjustments were required to reflect the new economic reality. The Reform Bill was very strongly supported in industrialized areas. West Indian slave owners, who relied on corrupt parliamentary boroughs, opposed it \endnote{\fullcite[vol. 1, p. 5.]{prentice}}. When the Reform Bill was first rejected, protests erupted. Eventually, the opponents relented, and the bill became law. England's political system reflected economic developments. The new Parliament was dominated by economic interests. The colonial commerce, which was formerly vital, had become less significant in the new capitalist society.