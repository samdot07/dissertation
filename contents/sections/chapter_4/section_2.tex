THE WEST INDIAN monopolists saw rapid industrial progress. They had the upper hand due to their status, established customs, previous economic contributions to Britain, and powerful location. However, they were unavoidably doomed \endnote{\fullcite[pp. 238-239.]{merivale}}. They battled tirelessly to retain their defective system, ignoring all other considerations. The resistance to the West Indians was not only about abolishing slavery, but also about destroying their monopoly. Their antagonists were both humanitarians and capitalists. The West Indian economic system was both morally wrong and financially unsustainable, making its demise inevitable \endnote{\fullcite[Add. MSS. 38295, f. 102. Anonymous to Lord Bexley. July, 1823.]{liverpoolpapers}. \parencite{ericwilliams}.}. As Hibbert pointed out, the colonies couldn't withstand their disdain for both slavery and monopolies \endnote{\fullcite[Lord Dudley and Ward, March 7, 1826. New Series, vol. 14, p. 1164]{hansardNS}. \parencite{ericwilliams}.}. This opposition occurred in three stages, the abolition of the slave trade in 1807, the abolition of slavery in 1833, and the repeal of sugar privileges in 1846. These events were related. The slave system's demise was ultimately caused by the same interests that built it. Humanitarians focused on the most vulnerable aspects of the system, which resonated with the people. This would not have been viable a century ago, when strong economic interests backed the colonial regime. The decline happened because capitalists abandoned slave owners and dealers. The West Indians, spoilt for 150 years, incorrectly felt their domination was natural rather than the result of mercantilism. They brought ancient principles into a new period of anti-imperialism. When Adam Smith's invisible hand turned against them, they had no other option than to seek supernatural intervention \endnote{\fullcite[Mr. Robinson, March 11, 1831. Third series, vol. 3, p. 354.]{hansardTS}. \parencite{ericwilliams}.}. The growth and collapse of mercantilism paralleled the rise and decline of slavery.
\subsection{protection or free trade}
If grain was the pinnacle of monopolies, sugar was the opposite. The challenge against preferential sugar tariffs in the West Indies was part of a larger campaign that had already ended the East India Company's monopoly in 1812 and overturned the Corn Laws in 1846. The treasurer of the Anti-Corn Law League stated that the organization was created on the same righteous principle as the Anti-Slavery Society, with the goal of allowing people to barter their work for as much food as possible \endnote{\fullcite[vol. 1, p. 75.]{prentice}}. The costly and unjust West Indian sugar monopoly became a focal point of the free trade movement. Supporters of East Indian sugar constantly chastised the West Indian monopoly, describing the islands as sterile rocks that devoured the nation's resources. By the end of the eighteenth century, Britain was prepared to abolish monopolies \endnote{\fullcite[pp. 17, 18–19, 26–27, 50–51, 53, 74–75.]{prinsep}}. East India's opposition grew stronger in the 1820s, with demands for equality with the West Indies rather than preferential treatment \endnote{\fullcite[Petition of merchants shipowners, etc., concerned in the trade to the East Indies, March 3, 1823. New Series, vol. 8, p. 339.]{hansardNS}. \parencite{ericwilliams}.}. They said that enjoying the monopoly for a long period did not warrant its continuance. West Indian planters said they deserved protection because of their investments in sugar growing \endnote{\fullcite[p. 37.]{macaulay}}, but this argument was disregarded as valid for any dangerous endeavor. Hume felt that British common sense and patriotism would oppose such monopolies since any constraints were detrimental \endnote{\fullcite[pp. 40-41.]{eastindiadebates}}. In 1820, London businessmen petitioned Parliament, claiming that freedom from restraint would stimulate overseas commerce and improve the economy \endnote{\fullcite[vol. 10, pp. 771-772.]{cambridgeMH}}. Mr. Finlay of Glasgow was an ardent supporter of free trade and the elimination of barriers. Liverpool businessmen also condemned trade monopolies, notably those of the East India Company, as damaging to national interests \endnote{\fullcite[May 16, 1820. New Series, vol. 1, pp. 424–425, 429.]{hansardNS}. \parencite{ericwilliams}.}. The West Indian monopoly was both logically unsound and unprofitable. It was time to abandon this antiquated policy that favored the rotten cause of West Indian slaveholders \endnote{\fullcite[p. 89.]{seely}}. To cut salaries, capitalists popularized the concept of the free breakfast table. Protective food obligations were viewed as unfair and ridiculous \endnote{\fullcite[Villiers, April 5, 1841. Third series, vol. 57, p. 920.]{hansardTS}. \parencite{ericwilliams}.}. Monopolies were not only costly and destructive, but they also shattered previous colonial empires \endnote{\fullcite[Labouchere, March 12, 1841. Third series, vol. 57, pp. 162–163.]{hansardTS}. \parencite{ericwilliams}.}. The West Indian interest was doomed. The defensive system was compared to monkeys stealing from one another and losing as much as they took \endnote{\fullcite[Villiers, April 5, 1841. Third series, vol. 57, p. 920.]{hansardTS}. \parencite{ericwilliams}.}. Ricardo encouraged planters to accept change, stating that the ball was rolling and could not be stopped \endnote{\fullcite[February 24, 1845. Third series, vol. 57, p. 1078.]{hansardTS}. \parencite{ericwilliams}.}. Leading statesmen had previously backed the West Indians, but Palmerston now sides with their opponents. The phrase protection should be eliminated from business jargon \endnote{\fullcite[p. 30.]{guedalla}} since it was detrimental to the country's success \endnote{\fullcite[May 31, 1850. Third series, vol. 111, p. 592.]{hansardTS}. \parencite{ericwilliams}.}. The West Indians were supported by protectionists, as well as the maize and sugar landed gentry. Bentinck, Stanley, and Disraeli were prominent personalities who championed the West Indian cause. However, Disraeli attacked it, claiming that they couldn't cling to relics of the protectionist system following huge developments in free trade \endnote{\fullcite[March 3, 1853. Third series, vol. 124, p. 1036.]{hansardTS}. \parencite{ericwilliams}.}. Mercantilism was considered outdated and antiquated. The West Indians opposed the free trade movement, claiming that the colonial arrangement ensured reciprocal monopolies \endnote{\fullcite[p. 208.]{penson}}. They saw their exclusive access to the home market as a just recompense for colonial constraints \endnote{\fullcite[p. 27.]{fletcher}. \fullcite[p. 30.]{memorandum}. \parencite{ericwilliams}. \fullcite[C.O. 137/140.]{reportjamaica}. \parencite{ericwilliams}.}. They said they required protection because of their competitors' advantages, citing low labor costs and fertile land in India \endnote{\fullcite[C.O. 137/140.]{reportjamaica}. \parencite{ericwilliams}.} and Brazil. Despite their repeated requests for protection, their arguments were rejected with contempt \endnote{\fullcite[Mr. Fitzgerald, March 18, 1831. Third series, vol. 3, p. 537]{hansardTS}. \parencite{ericwilliams}. \fullcite[Henry Goulburn, May 30, 1833. Third series, vol. 18, p. 111]{hansardTS}. \parencite{ericwilliams}.}. Before 1833, as slave owners, they wanted protection from free-grown sugar from India. After 1833, as employers of free labor, they sought protection from slave-grown sugar from Brazil and Cuba. They adapted their reasoning to meet their demands, but they always aimed to retain their monopoly. In the end, the West Indians remained shortsighted, wanting antiquated rights in a contemporary empire. Despite the transition from slavery to freedom, their expectations remained the same. They maintained pushing for additional monopolies to solve issues caused by monopolies while disregarding new realities \endnote{\fullcite[p. 84.]{merivale}}. In 1846, their motto of protection and labor remained unchanged from 1746. They regarded protection as justice \endnote{\fullcite[Bentinck, July 10, 1848. Third series, vol. 100, p. 356.]{hansardTS}. \parencite{ericwilliams}.} and considered it un-English to deny \endnote{\fullcite[Stewart, June 3, 1844. Third series, vol. 75, p. 213l.]{hansardTS}. \parencite{ericwilliams}. \fullcite[Miles, June 23, 1848. Third Series, vol. 99, p. 1094.]{hansardTS}. \parencite{ericwilliams}.}, believing it was necessary for free labor to succeed \endnote{\fullcite[Viscount Sandon, February 12, 1841. Third series, vol. 56, p. 616.]{hansardTS}. \parencite{ericwilliams}.}. Gladstone, a big supporter, attempted to defend the West Indian monopoly by arguing that it protected free-grown sugar from slave-grown sugar. However, he conceded that the distinction was ambiguous \endnote{\fullcite[February 26, 1845. Third series, vol. 77, p.1269.]{hansardTS}. \parencite{ericwilliams}.}, and the West Indian argument weakened after 1836, when protection was extended to East Indian sugar, which did not confront the same challenges \endnote{\fullcite[May 31, 1850. Third series, vol. 111, p. 581l.]{hansardTS}. \parencite{ericwilliams}.}. Protection cannot remain forever, and even twenty more years will not make West Indian farming healthy and sustainable \endnote{\fullcite[June 3, 1844. Third series, vol. 75, p. 198l.]{hansardTS}. \parencite{ericwilliams}.}.
\subsection{anti-imperialism}
The colonial system served as the foundation for commercial capitalism during the mercantile age. However, throughout the free trade era, industrial capitalists had little interest in keeping colonies, particularly those in the West Indies. This shift in attitude began in the early years of the Industrial Revolution and continued with the free trade movement. The entire world became similar to a British colony, resulting in the downfall of the West Indies. Cobden, a significant actor in this movement, praised Adam Smith's ideas on the economic burden of colonies in his famous work \endnote{\fullcite[April 10, 1851. Third series, vol. 115, p. 1440.]{hansardTS}. \parencite{ericwilliams}.}. For Cobden, the issue of colonies was largely monetary \endnote{\fullcite[April 10, 1851. Third series, vol. 115, p. 1443.]{hansardTS}. \parencite{ericwilliams}.}. He saw colonies as costly burdens that appealed to public emotion but ultimately raised government spending without improving the trade balance. Cobden considered colonialism as a bad policy move that harmed prospective trade with new markets \endnote{\fullcite[pp. 12-14.]{cobden}}. Molesworth, a prominent colonial reformer, said that Britain's colonial policies were motivated by a mad desire for a worthless empire \endnote{{\fullcite[June 26, 1849. Third series, vol. 106, pp. 942, 951–952, 958.]{hansardTS}}. \parencite{ericwilliams}. \fullcite[July 25, 1848. Third series, vol. 100, p. 825.]{hansardTS}. \parencite{ericwilliams}.}. The expense of defending this empire drained one-third of Britain's export commerce with the colonies, making colonial independence a more cost-effective alternative. Radical politician Hume also questioned the concept of Mr. Mother Country, pushing for the elimination of tight regulations over the colonies \endnote{\fullcite[February 3, 1830. New Series, vol. 22, p. 855.]{hansardNS}. \parencite{ericwilliams}.}, enabling them to run their own affairs rather than being ruled by Downing Street's changing policies \endnote{\fullcite[March 23, 1832. Third series, vol. 11, p. 834.]{hansardTS}. \parencite{ericwilliams}.}. The Colonial Office was regarded as an antiquated annoyance \endnote{\fullcite[June 19, 1848. Third series, vol. 99, p. 875.]{hansardTS}. \parencite{ericwilliams}.}. Roebuck, a freelance Radical, opposed the humanitarian unwillingness to offer colonies local autonomy \endnote{\fullcite[p. 286.]{morrell}}. Capitalists like Taylor, who were linked with the Colonial Office, saw colonies as chaotic entities ruled by inept governors, missionaries, and slaves \endnote{\fullcite[Introduction, pp. xiii, xxiv.]{bellmorrell}}. Merivale stated that colonies were preserved solely for the pleasure of governing them \endnote{\fullcite[p. 78.]{merivale}}. There was widespread agreement that the West Indies were doomed to failure. Desperate planters believed that there was a coordinated campaign to destroy the colonies \endnote{\fullcite[p. 22.]{jamaicaHoA}. \parencite{ericwilliams}.}, with certain members of Parliament eager to surrender the West Indies to America for low pay \endnote{\fullcite[Robinson, February 3, 1848. Third series, vol. 96, p. 75.]{hansardTS}. \parencite{ericwilliams}.}. These unproductive colonies have only caused war and expenditures \endnote{\fullcite[June 3, 1842. Third series, vol. 63, p. 1218–1219.]{hansardTS}. \parencite{ericwilliams}.}. They were always seen as the most onerous areas of the British Empire, and their disappearance would not reduce Britain's strength, riches, or influence \endnote{\fullcite[June 10, 1844. Third series, vol. 75, p. 462.]{hansardTS}. \parencite{ericwilliams}.}. This idea was widely held. Even Disraeli, a fervent imperialist in later life, shared this attitude. In 1846, he saw the forlorn Antilles as an essential component of England's colonial system \endnote{\fullcite[July 28, 1846. Third series, vol. 88, p. 164.]{hansardTS}. \parencite{ericwilliams}.}. However, six years later, Canada had become a diplomatic concern, with the colonies viewed as a burden \endnote{\fullcite[p. 351.]{woodward}. \fullcite[p. 519.]{morrell}. \fullcite[Introduction, p. XXVI.]{bellmorrell}}. Gladstone remarked that legislative attention to colonial affairs was infrequent and limited to party politics \endnote{\fullcite[vol. 1, p. 268.]{morley}}. The period of empire was over, replaced by the age of free merchants, economists, and calculators, signaling the end of the West Indies' prominence. However, after thirty years, opinions would alter still more.
\subsection{world sugar production}
Before 1783, the British sugar islands prospered because there was little competition in sugar production and they purposefully avoided any competitors. Their major competitors were Brazil and the French islands, while Cuba suffered constraints owing to Spanish mercantilist policies. Following the American colonies' freedom, Saint Domingue raced ahead. The cultivation in Barbados and Jamaica switched the sugar trade from Portugal to England, while the emergence of Saint Dominic allowed France to dominate the European sugar market. Saint Domingue, being larger and more productive than any British colony, had reduced production costs, which drew the attention of the Privy Council Committee in 1788. For British Prime Minister William Pitt, the economic benefit was critical. The age of the British sugar islands came to an end when their system became unprofitable, dependent on the slave trade, which Pitt saw as harmful to Britain's interests \endnote{\fullcite[April 2, 1792. vol. 29, p. 1147.]{parlhist}. \parencite{ericwilliams}.}. This represented a remarkable turn for Pitt, whose predecessors had backed West Indian interests and opposed abolition only a decade before. Pitt shifted his emphasis to India, hoping to reclaim control of the European sugar market with Indian sugar \endnote{\fullcite[p. 211.]{ragatzFP}.} while also lobbying for the global elimination of the slave trade \endnote{\fullcite[G.D. 8/102. Pitt to Eden, December 7, 1787.]{chathampapers}. \parencite{ericwilliams}. \fullcite[Pitt to Eden. January 7, 1788. vol. 1, p. 304.]{aucklandjournal}. \fullcite[Pitt to Grenville. August 29, 1788. vol. 1, p. 353.]{fortescue}}, which would undercut Saint Domingue. Pitt's scheme, however, failed due to high non-West Indian sugar taxes \endnote{\fullcite[pp. 213-214.]{ragatzFP}} and France, the Netherlands, and Spain's unwillingness to prohibit the slave trade \endnote{\fullcite[Add. MSS. 38224, f. 118.]{liverpoolpapers}. \parencite{ericwilliams}. \fullcite[Harris to Grenville, January 4, 1788. vol. 3, pp. 412–443]{fortescue}}. Gaston-Martin, a French historian, accused Pitt of exploiting humanitarian concerns to harm French industry \endnote{\fullcite[vol. 3, pp. 25, 39.]{gastonmartin} \parencite{ericwilliams}.}. The French Revolution presented Pitt with an unexpected chance. Fearing that the revolution would result in the abolition of slavery, the planters of Saint Domingue offered the island to England in 1791 \endnote{\fullcite[G.D. 8/349. West Indian Islands, Papers relating to Jamaica and St. Domingo. F.O. 27/36. Gower to Grenville.]{chathampapers}. \parencite{ericwilliams}.}, followed by those of the Windward Islands \endnote{\fullcite[F.O. 27/40, De Curt to Hawkesbury. December 18, 1792. Add. MSS. 38228, f. 197. January 3, 1793.]{liverpoolpapers}. \parencite{ericwilliams}.}. Pitt accepted the offer when war broke out with France in 1793, despite fruitless attempts to take Saint Domingue, which harmed Britain's military efforts. Accepting Saint Domingue required Pitt to forsake his public support for abolition, though he continued to campaign for it while acknowledging the practical realities of the slave trade. Under Pitt, Britain's role in the slave trade expanded \endnote{\fullcite[p. 116.]{klingberg} \parencite{ericwilliams}.}, and the country gained two new sugar possessions, Trinidad and British Guiana. Some historians ascribe Pitt's actions to a dread of revolutionary ideals, but his decisions were largely motivated by political concerns. His purpose was to undermine Saint Domingue, either by flooding Europe with cheaper Indian sugar or by seizing the island to restore Caribbean equilibrium. After Saint Domingue fell to France, the slave trade became a humanitarian issue. Its devastation brought to the end of the French sugar trade but did not save the British West Indies. New competitors appeared, with Cuba filling the void left by Saint Domingue and Bonaparte pushing beet sugar, resulting in a conflict between cane and beet sugar. Cuban sugar found a market in Europe under the American flag, whilst British West Indian surpluses accumulated in England, resulting in bankruptcy. In 1807, a parliamentary inquiry discovered that British West Indian planters were losing money owing to unfavourable international market conditions \endnote{\fullcite[H. of C. sess. Ap. pp. 4-6.]{reportCS}. \parencite{ericwilliams}. \fullcite[Hibbert, March 12, 1807. vol. 9, p. 98.]{hansard}. \parencite{ericwilliams}.}. Production needed to be lowered, hence the slave trade had to be abolished. Older colonies backed abolition because they were overpopulated with slaves \endnote{\fullcite[Hibbert, February 23, 1807. vol. 8, p. 985.]{hansard}. \parencite{ericwilliams}.}, as opposed to newer colonies that needed labor. The tension between established and upstart planters led to the abolitionist movement. Wilberforce applauded abolition \endnote{\fullcite[March 12, 1807. vol. 9, p. 101.]{hansard}. \parencite{ericwilliams}.}, not realizing it was the outcome of economic misery. Merivale maintained that abolition was only a temporary solution, pointing out that without fresh slaves, the West Indies could not compete in the more competitive nineteenth-century market \endnote{\fullcite[pp. 303, 313-317.]{merivale}}. By the end of the Napoleonic Wars in 1815, sugar planters had not fared better. India remained a powerful opponent, but new players such as Mauritius, Cuba, and Brazil arose. Sugar farming spread to Louisiana, Australia, Hawaii, and Java. Beet sugar's success resulted in the freedom of French colonial slaves in 1848, as well as its acceptance throughout Europe and America as a means of economic independence. Between 1793 and 1833, Britain's sugar imports quadrupled, thanks mostly to depleted soils in the older colonies, while output grew in the younger colonies \endnote{\fullcite[p. 20.]{ragatzStat}}. Britain relied on foreign sources for basic minerals. Large farms were more efficient, but their transportation was restricted. Jamaica, which had enormous plantations in the eighteenth century, struggled after liberation owing to labor shortages and rising wages, unable to compete with Cuba's superior infrastructure and lush land \endnote{\fullcite[vol. 1, p. 59.]{pezuela}}. The British West Indies lost their sugar monopoly, unable to compete with larger, more productive territories that still used slave labor. Overproduction led the British government to support West Indian farmers in order to compete with Brazilian and Cuban sugar, a move that capitalists strongly opposed. Overproduction resulted in abolition in 1807, followed by liberation in 1833. Britain first protested about West Indian underproduction, then about overproduction. Emancipation was viewed as a deliberate step in Parliament to provide planters a real monopoly of the domestic market by aligning output with demand. Efforts were undertaken to raise the expense of West Indian agriculture in order to end the monopoly and allow Britain to trade sugar worldwide. In 1836, East India sugar was accepted on equal terms, and in 1846, sugar tariffs were equalized. The British West Indian colonies fell out of favor until the Panama Canal and worker revolts brought them back into spotlight.