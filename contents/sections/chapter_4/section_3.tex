IN THE EIGHTEENTH CENTURY, significant interests in England collaborated to defend monopolies and the colonial system. However, after 1783, these same interests turned against monopolies and the West Indian slave system. British exports were mostly manufactured items, which could only be paid for with raw resources. Britain's export expansion was dependent on its capacity to accept raw commodities as payment. The British West Indian monopoly, which prohibited the import of sugar from non-British estates for local use, was a hurdle. Cotton producers, shipowners, and sugar refiners banded together, as did major industrial and commercial centers like as London, Manchester, Liverpool, Birmingham, Sheffield, and the West Riding of Yorkshire, to oppose West Indian enslavement and monopoly. Abolitionists concentrated their efforts on these industrial hubs \endnote{\fullcite[Cropper to Sturge. October 14, 1825.p. 84.]{henry}}.
\subsection{cotton}
In the eighteenth century, West Indian planters played a declining role in raw cotton exports and imports. The steam engine and cotton gin shifted Manchester's stance from indifferent to hostile. By 1788, Wilberforce had recognized Manchester's financial support for abolition, despite the city's extensive role in African commerce \endnote{\fullcite[Add. MSS. 34427, ff. 401–402 (v). Wilberforce to Eden. January 1788.]{aucklandpapers}. \parencite{ericwilliams}.}. Manchester had no representation in Parliament prior to 1832, therefore its explicit resistance to the West Indian system emerged later, despite earlier concerns. In 1830, Cobbett, a worker's rights campaigner, contested for Manchester's parliament seat. His hostility to landlords gained traction, but his views on West Indian slavery were controversial. He attacked Wilberforce and the Methodists and saw slaves as fat and lazy \endnote{\fullcite[p. 106.]{bready}}, which caused Manchester to reject him \endnote{\fullcite[vol. 1, pp. 3-4.]{prentice}}. Manchester favoured East Indian sugar over West Indian sugar. On May 4, 1821, the Manchester Chamber of Commerce petitioned Parliament to stop favoring slave colonies over free ones \endnote{\fullcite[p. 146.]{theodore} \parencite{ericwilliams}.}. In 1833, the city lobbied for the importation of Brazilian sugar for processing. Mark Philips, who represents Manchester in Parliament, emphasized the significance of this issue to the city's cotton sector \endnote{\fullcite[March 6, 1833. Third series, vol. 16, p. 290.]{hansardTS}. \parencite{ericwilliams}.}. Philips' company had historical links to the West Indian trade, but by 1832, he was against slavery and focused on removing monopolies. After 1833, Manchester's businessmen favored unrestricted commerce in sugar, even if it was produced by slaves. Philips advocated for equalizing East Indian sugar tariffs, arguing that planters should not get greater pay \endnote{\fullcite[April 29, 1836. Third series, vol. 33, p. 472.]{hansardTS}. \parencite{ericwilliams}.}. In 1839, he proposed identical taxes on all foreign sugar in order to reduce expenses and increase commerce with Brazil \endnote{\fullcite[June 28, 1839. Third series, vol. 48, p. 1029.]{hansardTS}. \parencite{ericwilliams}.}. John Bright and Milner Gibson advocated for free trade, describing protective levies on West Indian sugar as a obnoxious tax \endnote{\fullcite[Milner Gibson, February 24, 1845. Third series, vol. 77, p. 1053.]{hansardTS}. \parencite{ericwilliams}.} that raised expenses for British workers \endnote{\fullcite[Milner Gibson, July 3, 1848. Third series, vol. 100, p. 54.]{hansardTS}. \parencite{ericwilliams}.}. Bright contended that protection made planters chronic complainers \endnote{\fullcite[February 24, 1845. Third series, vol. 77, p. 1144.]{hansardTS}. \parencite{ericwilliams}. \fullcite[June 30, 1848. Third series, vol. 99, p. 1428.]{hansardTS}. \parencite{ericwilliams}.} and that cotton producers did not require protection \endnote{\fullcite[Bentinck, July 10, 1848. Third series, vol. 100, p. 324.]{hansardTS}. \parencite{ericwilliams}.}, dismissing previous and future requests for it. He recommended planters to grow alternate crops such as cloves and nutmeg \endnote{\fullcite[June 16, 1848. Third series, vol. 99, p. 747.]{hansardTS}. \parencite{ericwilliams}.}.
\subsection{iron}
In 1788, an organization devoted to eradicating slavery was formed in Birmingham, and money was raised to assist the cause \endnote{\fullcite[Add. MSS. 34427, ff 401–402 (v). Wilberforce to Eden. January, 1788.]{aucklandpapers}. \parencite{ericwilliams}.}. Prominent ironmasters, like Samuel Garbett, were extensively involved \endnote{\fullcite[vol. 1, p. 434.]{langford}}. Garbett, a key participant in the Industrial Revolution, worked with Roebuck at the Carron Works and invested with Boulton and Watt in the Albion Mills and Cornwall copper mines. His influence went beyond administration into industrial politics, making him a crucial advocate for the ironmasters before the government \endnote{\fullcite[p. 233.]{ashton}}. Garbett was a fierce antagonist who mostly represented Birmingham's industrial sector. On January 28, 1788, a group of prominent Birmingham residents, led by Garbett, decided to petition Parliament. The petition voiced disgust for trade based on violence and cruelty, underlining their commitment to the nation's commercial wellbeing \endnote{\fullcite[vol. 1, pp. 436, 440.]{langford}}. However, there was no complete agreement in Birmingham on the abolition question some slave trade manufacturers convened opposing meetings and sent counter-petitions to Parliament \endnote{\fullcite[vol. 1, p. 437.]{langford}}. West Indian interests viewed Garbett and his associates as opponents. Birmingham became the focus of political upheaval in 1832, led by ironmaster Attwood, which almost ignited a revolution and resulted in the Reform Bill of 1832. The community was again divided on the topic of emancipation. Birmingham and other industrial cities voted in 1833 for a shortened apprenticeship period under the Emancipation Act, which maintained a modified form of slavery. This mirrored a transition in the nineteenth century, when new interests emerged to oppose the colonial system. Sheffield, a steel manufacturing city, had little interest in colonialism and was pro-abolition. Sheffield, like Manchester and Birmingham, had no parliamentary representation until 1832 and was part of Yorkshire, represented by abolitionists Wilberforce and Brougham \endnote{\fullcite[pp. 4-5.]{norman}}. Sheffield's support for abolition arose in part from its concentration on the East. In 1825, abolitionists launched a boycott against West Indian commodities in favor of Indian sugar and rum, with Sheffield leading the charge. Sheffield's Anti-Slavery Society issued a memorial to the Prime Minister in May 1833, requesting quick liberation \endnote{\fullcite[p. 6.]{norman}} but opposing slave owner compensation and the apprenticeship program. Ultimately, Sheffield, like Birmingham, opted for the lowest apprenticeship time \endnote{\fullcite[July 25, 1833. Third series, vol. 19, p. 1270.]{hansardTS}. \parencite{ericwilliams}.}.
\subsection{wool}
The wool sector also expressed objection. In 1833, Mr. Strickland of Yorkshire questioned whether the House should favor free trade and economic expansion over strengthening monopolies through regulations. He decided that all monopolies should be destroyed since they impede commerce \endnote{\fullcite[March 6, 1833. Third series, vol. 16, p. 288.]{hansardTS}. \fullcite[June 17, 1833. Third series, vol. 18, p. 911.]{hansardTS}. \parencite{ericwilliams}.}. This attitude was supported by influential leaders such as John Bright in cotton and Samuel Garbett in iron, as well as Richard Cobden, a notable supporter for the wool sector. Cobden, a leader in the Anti-Corn Law League and an outspoken supporter of free trade, was passionately opposed to the West Indian monopoly. He called their claim to a monopoly audacious. Cobden claims that no member of Parliament would have proposed a monopoly in the past \endnote{\fullcite[June 10, 1844. Third series, vol. 75, pp. 446–447.]{hansardTS}. \parencite{ericwilliams}.}. He contended that if Britain had exchanged its exports for free trade with Brazil and Cuba, she would have profited \endnote{\fullcite[Third series, vol. 62, p. 1173.]{hansardTS}. \parencite{ericwilliams}.}. He questioned the nature of such trading, comparing the House of Commons' attitude to that of a badly run business \endnote{\fullcite[June 22, 1843. Third series, vol. 70, p. 210.]{hansardTS}. \parencite{ericwilliams}.}. Cobden dismissed the claim that the differential levy on West Indian sugar was intended to discourage the use of slave-grown sugar. He questioned the morality of a society that exported textiles made of slave-grown cotton to Brazil but denied slave-grown sugar in exchange \endnote{\fullcite[pp. 91-92.]{cobdenS}}. Cobden's arguments were convincing, if not generally accepted on moral grounds. The Anti-Slavery Party endorsed his position, citing its roots in industrial communities and its alliance with the Corn Laws repeal movement \endnote{\fullcite[February 24, 1845. Third series, vol. 77, p. 1128.]{hansardTS}. \parencite{ericwilliams}.}.
\subsection{Liverpool and Glasgow}
One of the most terrible realities for West Indians was Liverpool's move away from their interests. As of 1807, Liverpool still had seventy-two active slave traffickers. The final English slave dealer, Captain Hugh Crow, left Liverpool just before slavery was abolished \endnote{\fullcite[p. 283.]{averil}}. Despite characters like Tarleton opposing abolition in Parliament, Liverpool also had William Roscoe, who was recognized for his anti-slavery views. Liverpool's slave trade persisted in 1807, although it was becoming less important to the city's economy. Abolition did not ruin Liverpool \endnote{\fullcite[February 23, 1807. vol. 8, pp. 961–962.]{hansard}. \parencite{ericwilliams}.}. Rather, it shifted the focus from West Indian slavery and sugar to American slavery and cotton. The American cotton trade became Liverpool's most important business \endnote{\fullcite[pp. 31-32.]{buck}}. After assisting Manchester's expansion in the eighteenth century, Liverpool followed Manchester's lead in the nineteenth century age of laissez-faire and free commerce. After 1807, Liverpool elected politicians like Canning and Huskisson, who advocated for free trade but in a more muted tone. That same year, merchants and shipowners petitioned Parliament to review regulations that granted exclusive colonial trade monopolies \endnote{\fullcite[June 17, 1833. Third series, vol. 18, pp. 909–910]{hansardTS}. \parencite{ericwilliams}.}. They also encouraged Parliament to prioritize the welfare of local laborers alongside legislation that benefits slaves in faraway colonies \endnote{\fullcite[June 17, 1833. Third series, vol. 18, p. 910.]{hansardTS}. \parencite{ericwilliams}.}. In Glasgow, another city where West Indians formerly garnered support from personalities such as Macdowall and affluent sugar heiresses, events were shifting.
\subsection{sugar}
In the nineteenth century, like in the eighteenth, Britain sought to control the worldwide sugar trade, sweetening tea and coffee around the world in the same way that it had done during the Industrial Revolution. However, this objective was at odds with the waning importance of West Indian sugar production in comparison to other regions, as well as West Indian planters' efforts to cut down planting in order to keep prices high. The slave insurrection in Saint Domingo led sugar prices in Europe to rise \endnote{\fullcite[p. 9.]{ragatzStat}}. By 1792, English sugar refiners, who had previously been more cautious, petitioned Parliament, blaming their industry's troubles on the West Indian monopoly. They wanted greater charges on foreign sugar delivered by British ships, as well as equal levies on East and West Indian sugars \endnote{\fullcite[pp. 3, 8, 15.]{reportPR}}. The public mistakenly blamed refiners for high prices \endnote{\fullcite[p. 18n.]{reportPR}}, but a committee at a public hearing cleared them and advocated fair conditions for East Indian sugar imports as a matter of fairness \endnote{\fullcite[Add. MSS. 38227, 217. Chairman to Hawkesbury. January 23, 1792. ff. 219222. Chairman to Pitt. January 12, 1792.]{liverpoolpapers}. \parencite{ericwilliams}.}. The Indian sugar issue revived in the 1820s, when India attempted to sell products to pay for British purchases. Faced with competition from American cotton \endnote{\fullcite[vol. 5-7, 11.]{customs5}. \parencite{ericwilliams}. \fullcite[p. 308.]{baines}}, Indian businessmen pushed for entrance to Britain's sugar market \endnote{\fullcite[p. 144.]{theodore} \parencite{ericwilliams}.}. The refiners disagreed with the East Indians on free commerce and preferred to keep their monopoly. Ricardo called for unlimited sugar imports from all sources \endnote{\fullcite[p. 19.]{eastindiadebates}}. By 1831, British refiners were in severe circumstances, as West Indian sugar monopolized the domestic market. Parliament approved yearly acts permitting only Brazilian and Cuban sugar to be refined and re-exported, which was unacceptable to British business. The government, sympathetic to colonial interests, made a temporary compromise in 1832, confirming West Indian monopolies in exchange for freedom. Foreign sugar was permitted for refining and export, but limitations persisted, citing the slave labor employed in Brazilian and Cuban production. This difference was selective, since British industry relied on both slave-grown cotton and coffee. Critics pushed for open commerce, citing the high costs generated by monopolies. They saw sugar limitations differently than other items \endnote{\fullcite[William Ewart, April 3, 1833. Third series, vol. 17, p. 75.]{hansardTS}. \fullcite[William Ewart, May 10, 1841. Third series, vol. 58, p. 101.]{hansardTS}. \parencite{ericwilliams}.}. Ultimately, capitalists preferred cheap sugar and opposed monopolistic pricing. They challenged the government's strategy of prioritizing West Indian sugar over other sources \endnote{\fullcite[Benjamin Hawes, February 12, 1841. Third series, vol. 56, p.608.]{hansardTS}. \parencite{ericwilliams}.}. The fight over sugar trading reflected larger contradictions between free trade ideas and colonial economic interests.
\subsection{shipping}
The West Indians often justified their system by citing their assistance to England's naval dominance. However, Clarkson's careful study showed the full cost of his donation. Clarkson, risking his life, explored docks in Liverpool, Bristol, and London, interviewing sailors, checking muster registers, and gathering information that harshly denounced the slave trade's impact on both enslaved Africans and sailors. According to Clarkson, the slave trade had a fatality rate twenty times that of the Newfoundland trade \endnote{\fullcite{ramsay}.}. William Smith disproved the misconception that the slave trade produced many untrained sailors, landsmen, finding that only a tiny percentage of seamen recorded in Bristol and Liverpool were newcomers \endnote{\fullcite[vol. 29, p. 332.]{parlhist}. \parencite{ericwilliams}.}. According to the Abolition Committee, seafarers in the slave trade died more than twice as often than in any other area of English business \endnote{\fullcite[February 23, 1807. vol. 8, pp. 948–949.]{hansard}. \parencite{ericwilliams}.}. John Newton, a major scholar, called the death toll in the slave trade as truly alarming \endnote{\fullcite[p. 8.]{newton}}. Ramsay summed up the prevalent feeling that the trade did not generate sailors, but rather destroyed them, making a persuasive case for its abolishment \endnote{\fullcite{ramsay}.}. By 1807, shipowners had lost interest in the slave trade, deeming the West Indian monopoly costly. They were told that equalizing taxes on East Indian sugar would increase shipping employment \endnote{\fullcite[p. 56.]{reportLiv}}. Shipowners realized the importance of Brazilian sugar \endnote{\fullcite[September 28, 1831. Third series, vol. 7, p. 755.]{hansardTS}. \parencite{ericwilliams}.}. While lobbying for free trade, shipowners resisted giving up control of shipping monopolies, even after legislation changed in 1825 to allow the British West Indies to trade internationally. The Navigation Laws, which were crucial to England's colonial system, were repealed in 1848 amid a rising laissez-faire attitude, marking the end of an era. Shipowners opposed the repeal, citing their earlier opposition to grain and sugar monopolies while insisting on maintaining control of shipping. The removal of the Navigation Laws in 1848 dealt a last blow to mercantilism, as Ricardo urged proponents of the long voyage to sail their goods around the British Isles. \endnote{\fullcite[Introduction, p. xli.]{bellmorrell}}