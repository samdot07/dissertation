CAPITALISTS FIRST backed slavery in the West Indies before helping to abolish it. When British business was reliant on the West Indies, slavery was either overlooked or justified. However, when the West Indian monopoly proved to be a problem, British businessmen outlawed West Indian slavery in order to break it. This demonstrates that their opinion on slavery was situational, shifting depending on geographical and economic factors. This is obvious in their stance against slavery in Cuba, Brazil, and the United States after 1833, when they derided those who saw slavery solely in terms of sugar \endnote{\fullcite[G. Thompson, June 26, 1848. Third series, vol. 99, p. 1223.]{hansardTS}. \parencite{ericwilliams}.} and refused to establish tariffs on moral considerations or enforce anti-slavery measures at customs houses \endnote{\fullcite[Lord John Russell, June 3, 1844. Third series, vol. 75, p. 170.]{hansardTS}. \parencite{ericwilliams}.}. Between 1815 and 1833, the British government sought to pay Spain and Portugal to halt the slave trade, but these efforts were unsuccessful since abolition would affect the economies of Cuba and Brazil. Under pressure from the West Indians, Britain implemented more harsh sanctions. At the Verona conference, Wellington proposed a boycott of products from countries still involved in the slave trade, but this was met with opposition and viewed as motivated by self-interest rather than humanitarian concerns \endnote{\fullcite[Wellington to Canning. October 28, 1822. vol. 1, p. 329.]{wellington}}, as noted by a member of Parliament who later referred to it as lucrative humanity \endnote{\fullcite[Third series, vol. 96, p. 1096.]{hansardTS}. \parencite{ericwilliams}.}. Canning saw the independence of Brazil as an opportunity to negotiate abolition in exchange for recognition \endnote{\fullcite[Wellington to Canning. September 30, 1822. vol. 1, p. 329.]{wellington}}, but there was a possibility that France would recognize Brazil while continuing the slave trade \endnote{\fullcite[Memorandum for the Cabinet. November 15, 1822. vol. 1, p. 62.]{stapleton}}. Canning stressed to Wilberforce the importance of British interests in Brazilian commerce, emphasizing the necessity for care and a balanced approach that took into account both business and moral considerations \endnote{\fullcite[October 24, 1822. vol. 2, p. 466.]{wilberforce}}. British entrepreneurs rapidly clarified their priorities. A measure introduced in 1815 sought to prohibit the slave trade as an investment for British money, and by 1824, London businessmen were advocating for South American independence \endnote{\fullcite[June 15, 1824. New Series, vol. 11, p. 1345.]{hansardNS}. \parencite{ericwilliams}.}, emphasizing the economic benefits \endnote{\fullcite[June 23, 1824. New Series, vol. 11, p. 1475–1477.]{hansardNS}. \parencite{ericwilliams}.}. Smuggling was insufficient for British business, particularly in slave-labor countries such as Brazil. Capitalists strongly resisted the government's efforts to prevent the slave trade by military action on the African coast, claiming it was a waste of human life \endnote{\fullcite[June 15, 1830. New Series, vol. 25, p. 398.]{hansardNS}. \parencite{ericwilliams}.}. Prior to 1833, British entrepreneurs were heavily involved in the slave trade, purchasing slaves with British commodities. In 1845, Peel did not deny British involvement in the slave trade \endnote{\fullcite[Third series, vol. 96, p. 1095.]{hansardTS}. \parencite{ericwilliams}.}. British banks in Brazil sponsored slave dealers, while mining firms used slave labor. In 1843, John Bright, mindful of his constituents' interests, advocated against a law forbidding British wealth from participating in the slave trade, thinking that individual morality should prevail \endnote{\fullcite[August 18, 1843. Third series, vol. 71, p. 941.]{hansardTS}. \parencite{ericwilliams}.}. Capitalists were fed up with Britain's attempts to restrict the slave trade, believing that free commerce would eventually eradicate slavery \endnote{\fullcite[Ewart, February 24, 1845. Third series, vol. 77, p. 1066.]{hansardTS}. \parencite{ericwilliams}. \fullcite[June 22, 1843. Third series, vol. 70, p. 224.]{hansardTS}. \parencite{ericwilliams}. \fullcite[Hawes, June 23, 1848. Third series, vol. 99, p. 1121.]{hansardTS}. \parencite{ericwilliams}.} and contending that measures to prohibit the slave trade were expensive and inefficient \endnote{\fullcite[June 16, 1848. Third series, vol. 99, p. 748.]{hansardTS}. \parencite{ericwilliams}.}. Efforts to influence public opinion in slave-trading countries were seen more effective than coercion \endnote{\fullcite[Hutt, February 22, 1848. Third series, vol. 96, p. 1101.]{hansardTS}. \parencite{ericwilliams}.}. Britain's efforts have merely boosted the slave trade. Palmerston, a noted opponent of the slave trade, accomplished little while in power but urged the government to act more aggressively while out of office. His lectures were full of anti-slavery rhetoric, but the actual effects were limited \endnote{\fullcite[Peel, July 16, 1844. Third series, vol. 76, pp. 947, 963.]{hansardTS}. \parencite{ericwilliams}.}. Disraeli decried the economic consequences of ending the slave trade, but Wellington regarded it as a breach of international law \endnote{\fullcite[March 24, 1848. Third series, vol. 98, pp. 994–996.]{hansardTS}. \parencite{ericwilliams}.}. Gladstone originally supported repression in 1841, but subsequently criticized it as impracticable. Former West Indian slave owners turned abolitionists, after previously predicting economic ruin if the slave trade was abolished. By 1849, there was a large anti-slavery movement in Jamaica, with all socioeconomic strata and groups clamoring for justice in Africa \endnote{\fullcite[pp. 65, 94–95, 99, 120, 201, 249, 267.]{turnbull}}. Despite these developments, British entrepreneurs remained committed to profit. They had abolished West Indian slavery but continued to gain from it in Brazil, Cuba, and the United States. The West Indian monopoly was permanently ended.