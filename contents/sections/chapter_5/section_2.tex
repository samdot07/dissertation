THIS RESEARCH HAS purposefully minimized both the harshness of the slave system and the humanitarian efforts that led to its abolition. Ignoring these factors totally would be a grave historical error, missing out on one of the most major propaganda efforts of all time. Humanitarians were instrumental in destroying the West Indian slave system and rescuing the enslaved people. However, their importance is frequently misconstrued and exaggerated by people who stress emotion over scholarly precision, similar to the old scholars who prized religion over reason and proof. This misperception is due in part to contemporaneous actors' deliberate attempts to offer a biased perspective of the abolitionist movement. The slave trade was outlawed in 1807, and the statute said that it was contrary to the principles of justice, humanity, and sound policy. Lord Hawkesbury protested, saying that the phrases justice and humanity were used to harshly attack slave dealers. He offered an amendment to delete those terms, presenting abolition as a question of practicality rather than morality. The Lord Chancellor objected, claiming that the modification would undercut the basis for inviting other countries to join in abolition efforts. The Earl of Lauderdale felt that these lines were critical to the bill because their exclusion may lead France to believe that British abolition was motivated by selfish interests, considering that British territories were teeming with enslaved individuals \endnote{\fullcite[February 6, 1807. vol. 8, pp. 679–682.]{hansard}. \parencite{ericwilliams}.}. The British humanitarians were an impressive group. Clarkson reflected the finest of the era's humanitarian efforts. His acute knowledge of the injustice of slavery is relevant today. Clarkson was a relentless worker who carried out lengthy and perilous research on the slave trade's circumstances and consequences. He was also a prolific pamphleteer, and his history of the abolitionist movement is considered a classic. Clarkson was ardent and eager, often too much for his colleagues \endnote{\fullcite[Samuel Hoare to Clarkson, Add. MSS., 21254, ff. 12–12 (v).]{clarksonPr}. \parencite{ericwilliams}.}, but he was a rare and loyal supporter of the Negro race. James Stephen, both father and son, were important personalities. The elder Stephen, a lawyer in the West Indies, had direct knowledge of the situation. His son, the Colonial Office's first notable permanent under-secretary, was a staunch defender for enslaved people. He continually urged Wilberforce to conduct more public measures rather than holding private meetings and memorials. Stephen prepared the Emancipation Bill, grudgingly incorporating concessions for the planters \endnote{\fullcite[Minute of Stephen, September 15, 1841. p. 420.]{bellmorrell}}. Unlike others who got complacent, he remained vigilant about colonial regulations designed to safeguard the enslaved. James Ramsay, one of the first and most ardent abolitionists, had twenty years of experience with slavery as a rector in the West Indies. He was fully aware that the difficult conditions resulted in significant mortality rates among both white sailors and enslaved people \endnote{\fullcite[pp. 60-61.]{klingberg} \parencite{ericwilliams}.}. Planters chased Ramsay mercilessly. In comparison to these men, Wilberforce looked less noteworthy. He was an inept leader who preferred moderation, compromise, and delay. He avoided drastic measures and public uprisings, instead depending on aristocratic backing, legislative techniques, and private influence \endnote{\fullcite[p. 77.]{stephen}. \fullcite[p. 78.]{henry}}. Although a powerful and articulate speaker, Wilberforce was recognized for his saintly character and lack of personal benefit, which influenced Pitt's choice to have him lead the parliamentary struggle. Abolitionists, whom the planters referred to as visionaries and fanatics, were compared to hyenas and tigers \endnote{\fullcite[Wilberforce, June 15, 1824. New Series, vol. 11, p. 1413.]{hansardNS}. \parencite{ericwilliams}.}. Along with others such as Macaulay, Wesley, Thornton, and Brougham, they elevated anti-slavery views to near-religious significance in England, gaining the epithet the Saints. These reformers were not radicals, rather they were conservative on domestic problems. Methodists gave English workers Bibles instead of bread, while Wesleyan capitalists expressed open scorn for the working class. Many people incorrectly believe that abolitionists were always devoted to total emancipation. For a long time, they denied any intention of emancipating slaves, focused instead on ending the slave trade and believed that emancipation would ultimately come without legislative action. The Abolition Committee repeatedly denied any intention of liberation \endnote{\fullcite[Add. MSS. 21255, f. 50 (v). Add. MSS. 21256, ff. 40 (v), 96 (v).]{clarksonPr}. \parencite{ericwilliams}.}, and it wasn't until 1823 that emancipation became their proclaimed goal, owing mostly to the persecution of missionaries in the colonies. Even so, emancipation was meant to be gradual. Buxton advocated for gradual, quiet, and nearly imperceptible abolition of slavery, urging caution against hasty moves \endnote{\fullcite[May 15, 1823. New Series, vol. 9, pp. 265–266.]{hansardNS}. \parencite{ericwilliams}.}. The aim for a gradual abolition of slavery was not achieved in either England or the United States. In 1830, the July Revolution in France sparked legislative change in England. Although abolitionists continued to campaign and send petitions to ministers, colonial slavery and monopolies remained. Conservatives and radicals battled during a huge anti-slavery meeting in May 1830. A new policy evolved, pushing people to fight for liberation because the government would not act \endnote{\fullcite[March 28, 1833. pp. 101–102.]{henry}}. However, the abolitionist leadership exclusively criticized slavery among Negroes in the British West Indies, neglecting other places. They advocated a pious and silly crusade \endnote{\fullcite[Introduction, pp. xiv–xv.]{cochin}} against West Indian planters, demanding a boycott of slave-grown crops in favor of Indian items. This endeavor was sponsored by a variety of groups, including the Peckham Ladies' African Anti-Slavery Association, who felt that the boycott would successfully weaken slavery \endnote{\fullcite[p. 3.]{naish}}. The abolitionists also advocated for free-grown cotton \endnote{\fullcite{freecottonmovement}.}, expecting that it would aid in the eradication of American slavery \endnote{Gurney to Scoble, December 5, 1840, (Wilberforce Museum), D.B. 883?}. However, they ignored the fact that slavery existed in India, despite statutory vows to reduce it. This paradox damaged their humanitarian posture because they did not boycott slave-grown sugar from Brazil and Cuba. After 1833, they continued to attack West Indian planters who now utilized free labor while supporting Brazilian slave owners, demonstrating their contradictory viewpoint. Initially, abolitionists sought worldwide abolition of the slave trade. They promoted their cause at international conventions and were even prepared to go to war to prevent slavery from resuming \endnote{\fullcite[pp. 147-148.]{klingberg} \parencite{ericwilliams}.}. However, by 1833, they had adopted a more pacific posture, focused on the slave traffic rather than slavery itself. Buxton opposed the slave squadron's efforts for creating greater misery \endnote{\fullcite[Hutt, June 24, 1845. Third series, vol. 81, p. 1159.]{hansardTS}. \parencite{ericwilliams}.}, and Sturge rebuilt the Anti-Slavery Society as a nonviolent organization \endnote{\fullcite[Hutt, March 19, 1850. Third series, vol. 109, p. 1098.]{hansardTS}. \parencite{ericwilliams}.}. As time progressed, influential leaders such as Thomas Babington Macaulay advocated against utilizing trade policy to combat slavery, exposing discrepancies in the abolitionist cause \endnote{\fullcite[February 26, 1845. Third series, vol. 77, pp. 1290, 1292, 1300, 1302.]{hansardTS}. \parencite{ericwilliams}.}. Slavery began to be regarded as a necessary societal evil rather than a direct wrong \endnote{\fullcite[February 4, 1848. Third series, vol. 96, p. 85.]{hansardTS}. \parencite{ericwilliams}.}. Some, such as Disraeli and Carlyle, even saw emancipation as a mistake, with Carlyle proposing that the oppressed should be made to labor \endnote{\fullcite[February 4, 1848. Third series, vol. 96, p. 133.]{hansardTS}. \parencite{ericwilliams}. \fullcite{carlyle}.}. To summarize, the abolitionist movement's history was complex and frequently conflicting. While they played an important role in the abolition of the British slave trade and, subsequently, slavery, their actions and policies were not always consistent or compassionate.