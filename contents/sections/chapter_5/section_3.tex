WE INVESTIGATED the varying perspectives on slavery held by the British government, businessmen, absentee planters in the British West Indies, and humanitarians. We have monitored the anti-slavery movement within Britain. However, it is critical to remember that this was not only a domestic conflict in Britain. The fate of the colonies was at risk, and the colonists were highly aroused, reflecting and influencing the major events taking on in Britain. First, there were the white planters, who had to navigate relationships with both the British Parliament and the enslaved people. Then there were free individuals of color. Finally, there were the slaves themselves. Concerning the planters, in 1823, the British government implemented a program to reform slavery in the West Indies, which was enforced by orders in council in the Crown Colonies of Trinidad and British Guiana. The goal was that success here would inspire self-governing colonies to take similar steps willingly. This program sought to improve conditions rather than achieve rapid liberation, and it promoted gradual progress rather than abrupt upheaval. The objective was to put an end to slavery by compassionate treatment. Planters in both the Crown Colonies and the self-governing islands fiercely opposed this program, characterizing it as a mere catalogue of indulgencies to the Blacks \endnote{\fullcite[C.O. 28/95. House of Assembly, Barbados. November 15, 1825.]{co}. \parencite{ericwilliams}.} They anticipated that any compromises would result in further demands. None of the recommendations gained universal approval from West Indian planters. They were particularly angered by suggestions to remove the whipping of female slaves and the Negro Sunday market. They believed that punishing women was essential even in civilized cultures. The governor of Jamaica saw the planters' opposition to changing the Sunday market as so strong that he believed any attempt to change it was unwise, suggesting that it should be left to the operation of time and that change of circumstances and opinions which is slowly but surely leading to the improvement of the habits and manners of the slaves \endnote{\fullcite[C.O. 137/148. Manchester to Bathurst. July 10, 1819.]{co}. \parencite{ericwilliams}.}. The planters insisted that the whip was necessary for maintaining discipline. Without it, they declared, Adieu to all peace and comfort on plantations \endnote{\fullcite[C.O. 28/92, p. 24.]{reportDC}. \parencite{ericwilliams}.}. They opposed not only the British government's specific proposals, but also the imperial parliament's right to legislate on internal matters, describing these laws as arbitrary mandates... so positive and unqualified in point of matter, and so precise and peremptory in time \endnote{\fullcite[C.O. 28/99. Carrington, Agent for Barbados, to Bathurst. March 2, 1826.]{co}. \parencite{ericwilliams}.}. The conflict between slave owners discussing rights and freedoms was regarded as the clamour of ignorance. \endnote{\fullcite[C.O. 28/92, September 3. p. 33.]{reportWar}}. This was more than simply frantic chatter or a disdain for the imperial rulers' temperate but authoritative admonition \endnote{The phrase is Canning’s. \parencite{ericwilliams}.}. It was a message aimed not so much at the British public as at the slaves in the West Indies, who were unlikely to forget that, as the governor of Barbados put it, the love of power of these planters over the poor Negroes, each in his little sugar dominion, has found as great an obstacle to freedom as the law of their labor \endnote{\fullcite[C.O. 28/111. Smith to Stanley, July 13, 1833.]{co}. \parencite{ericwilliams}.}. Emancipation would occur despite, rather than because of, the planters. While whites plotted treachery and considered secession, free people of color remained loyal. While whites declined to run for office, mulattoes insisted on the right to public service \endnote{\fullcite[C.O. 295/92. Memorial for ourselves and in behalf of all our fellow subjects of African descent, William Clunes to Goderich. January 27, 1832.]{co}. \parencite{ericwilliams}.}. Their allegiance was motivated not by intrinsic virtue, but by a lack of power to secure their rights independently and a belief that there was no other way to achieve independence but via the British authority. Local governments had to rely on them when implementing anti-monopoly measures. In Barbados, the governor noted that the mulattoes had the advantage in terms of refinement, morals, education, and energy, but the whites had only old privileges and prejudices to back up their illiberal stance \endnote{\fullcite[C.O. 28/111. Smith to Stanley. May 23, 1833.]{co}. \parencite{ericwilliams}.}. Contrary to common assumption, as Britain's political crisis worsened, the most dynamic and strong social force in the colonies was the slave himself. This component of the West Indian issue has frequently been overlooked, as if slaves, once exploited as manufacturing tools, were not people. The planters considered slavery as everlasting, decreed by God, and used scripture to justify it. However, the slaves used the same texts for their own ends. They reacted to compulsion and punishment with lethargy, sabotage, and rebellion. Their most common form of opposition was passive, manifested as laziness. The assumption that Negro slaves were docile is a lie. The Maroons of Jamaica and the Bush Negroes of British Guiana were escaped slaves who signed contracts with the British government and lived autonomously in mountain or jungle retreats. They served as role models for slaves in the British West Indies in terms of achieving freedom. It seems impossible that economic crises and large-scale agitations in Britain would have no impact on slaves and their relationships with owners. Pressure on sugar plantations from British investors was exacerbated by pressure from slaves in the colonies. The slaves were not as dumb as their masters believed, and they were well aware of the talks concerning their fate. The planters freely discussed slavery in front of the persons whose fate was being determined. The planters were chastised for this, but they couldn't prevent it. This occurs throughout all severe societal crises. Before the French Revolution, the French court and nobility openly debated Voltaire and Rousseau, and in some cases, with genuine admiration. The planters' arrogant attitudes and harsh words enraged the already restless slaves.