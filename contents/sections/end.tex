This thesis examines Eric Williams' key work Capitalism and Slavery, which examines the fundamental links between the transatlantic slave trade, colonial exploitation, and the emergence of industrial capitalism in Europe, notably in Britain. Williams' claim that the wealth created by slavery and the colonial system had a key part in the emergence of Western industrial dominance defies conventional wisdom, which frequently separates the Industrial Revolution from colonial exploitation. By examining the economic, social, and political components of this connection, this study demonstrated how slavery has had a lasting influence on modern economies and evaluated Williams' connections between slavery and the rise of capitalism using primary materials from the transatlantic slave trade, plantation records, and historical economic data, as well as secondary literature. This approach enabled us to better comprehend the disputes around Williams' theory and give a complete examination of his contributions to history and economics. The thesis' structure, split in five parts, which mirrors Williams' book, has allowed for a full investigation of numerous elements of this historical relationship. Part one explored the origins and economic evolution of slavery, laying the groundwork for understanding its fundamental role in capitalist progress. Part two looked at the triangular trade and how it affected British industry and the West Indian economy, demonstrating deep economic interdependence. Part three examined the economic and political implications of the American Revolution on slavery and capitalism, emphasizing the revolutionary nature of this time. Part four looked at the larger growth of capitalism, the new industrial order, and economic dynamics in the West Indies, emphasizing the systemic aspect of these events. Finally, part five focused on the business sides of slavery, abolitionists' roles, and enslaved people's experiences, adding a human dimension to the economic aspect. Williams' work is still very important today, as modern arguments over reparations for slavery, economic injustice, and the legacy of colonialism repeat the issues he analyzed. The plea for reparations emphasizes the continued economic and social injustice caused by the historical exploitation of enslaved people \endnote{\fullcite{beckert2}. \fullcite{baptist}.}. Discussions regarding economic inequality and racial prejudice frequently bring up the long-term effects of slavery on wealth distribution and systematic racism \endnote{\fullcite{beckert}.}. The role of slavery in the development of global capitalism remains a key topic in academic research, encouraging us to rethink historical narratives and consider the ethical responsibilities of former colonial powers and modern corporations\endnote{\fullcite{drescher2}. \fullcite{sharon}.}. Understanding the historical relationship between capitalism and slavery is critical not just for academic research, but also for dealing with contemporary issues such as racial injustice, economic disparities, and colonialism's enduring impact. By analyzing Williams' arguments, this thesis contributes to the ongoing discussion of these important topics and encourages a better understanding of the historical roots of modern economic systems. Williams captures the most important point of his research by claiming that British and French capitalism established the foundation for contemporary industrial wealth, parliamentary democracy, and related liberties. His insights into the economic incentives underlying key historical upsets, such as the loss of American colonies and the following exploitation of Indian territories, demonstrate a pattern of economic interests driving political and social change. Williams also underlined that the crisis, which began in 1776 and lasted through the French Revolution, Napoleonic Wars, and the Reform Bill of 1832, was principally caused by economic issues, similar to today's global problems. He added that commercial capitalism of the eighteenth century, characterized by slavery and monopoly, boosted Europe's riches and set the way for industrial capitalism of the nineteenth century. This industrial capitalism eventually displaced commercial capitalism, slavery, and other behaviors. He emphasized the significance of understanding economic advances in order to make sense of historical events, as well as the dangers of viewing political and moral ideals in isolation from economic advancement. Reflecting on historical and present economic shifts, Williams suggested that the liberation of Africa and the Far East from imperialism will be decided by production needs, similar to how increased production power in 1833 transformed the mother-country-colony relationship. He stated that historians must learn from history in order to deliver important insights rather than simply as cultural decoration or pleasure. \endnote{\fullcite[pp. 167-169.]{ericwilliams}}.By combining these ideas with the today's  ongoing arguments, we highlight Williams' work's long-term value in comprehending the economic and moral forces that form our world. As we struggle with concerns of inequality and fairness, Williams' strong arguments force us to confront painful realities about our economic past and their long-term consequences for modern society. This thesis sought to disclose Williams' study connect historical insights with contemporary debates, illustrating the persistent significance of his innovative research in today's ongoing concerns about economic fairness and historical responsibility.