CAPITALISM AND SLAVERY, written by Eric Williams and published in 1944 \endnote{\fullcite{ericwilliams}.}, is a key work in economic history. Williams, a historian from Trinidad who later became the first Prime Minister of Trinidad and Tobago, argues that the transatlantic slave trade and the exploitation of enslaved Africans were crucial to the rise of industrial capitalism in Europe, especially in Britain. He challenges the common belief that the Industrial Revolution was separate from colonial exploitation. Instead, he claims that the profits from slavery funded the industrialization of Britain. He also argues that the abolition of slavery was driven more by economic interests than by moral reasons. In this thesis, I will explore Williams' argument that the wealth generated from slavery and colonialism significantly helped develop capitalism in Europe. By looking at the economic, social, and political aspects of this relationship, I aim to highlight the lasting impact of slavery on modern economic systems and challenge traditional histories that downplay the importance of slavery in the rise of Western industrial power. Williams' work is still relevant today because it forces us to reconsider the roots of modern economic systems and the ongoing inequalities caused by historical exploitation. Understanding the connection between capitalism and slavery is important not only for historical research but also for addressing current issues like racial injustice, economic disparity, and the legacy of colonialism. This thesis aims to add to this ongoing discussion by providing a detailed analysis of Williams' arguments and their implications for today's economic and social issues. Several present-day debates underscore the relevance of Williams' thesis. One such debate involves the call for reparations for slavery \endnote{\fullcite{coates}.}, with advocates arguing that the wealth generated from slavery significantly contributed to Western economies, necessitating compensation to address historical injustices \endnote{\fullcite{darity}.}. Similarly, discussions about economic inequality and racial disparities often cite the enduring legacy of slavery in contributing to wealth gaps and systemic racism in modern societies \endnote{\fullcite{baptist}.}. Furthermore, the role of slavery in the development of global capitalism remains a critical area of academic inquiry, prompting historians and economists to reassess the contributions of colonial exploitation to modern economic systems \endnote{\fullcite{beckert}.}. This ties into broader debates on the impacts of colonialism and the ethical considerations for former colonial powers and corporations regarding their historical accountability and responsibilities \endnote{\fullcite{blight}.}. Finally, Williams' work challenges the way slavery is remembered and taught, sparking discussions on historiography and public memory. These debates underscore the ongoing relevance of Capitalism and Slavery and its implications for understanding the complex legacies of slavery and capitalism in today's world. Since the publication of Capitalism and Slavery, many scholars have discussed Williams' ideas, either supporting or criticizing them. Historians like Seymour Drescher \endnote{\fullcite{drescher}.} and David Eltis \endnote{\fullcite{eltis}.} have questioned parts of Williams' economic arguments, while others, like Barbara Fields \endnote{\fullcite{fields}.} and Robin Blackburn \endnote{\fullcite{blackburn}.}, have expanded on his ideas, exploring the complex relationship between capitalism and slavery. This body of literature provides a solid foundation for this thesis, which seeks to navigate these debates and offer a comprehensive analysis of Williams' contributions to historical and economic scholarship. By looking at primary sources from the transatlantic slave trade, plantation records, and economic data from the period, along with a review of secondary literature, this study will evaluate the connections Williams draws between slavery and the development of capitalism. The thesis is structured into five parts, following the structure of Williams' book. Part one looks at the history and economic reasons for the development of slavery. Part two focuses on the triangular trade and its impact on British industry and the West Indies economy. Part three examines the economic and political effects of the American Revolution in relation to slavery and capitalism. Part four explores the broader development of capitalism, the new industrial order, and the economic dynamics involving the West Indies. Part five analyzes the commercial aspects of slavery, the role of abolitionists, and the experiences of the enslaved people. The conclusion will summarize the findings from these five parts and discuss the relevance of Williams' thesis today, especially in addressing historical injustices in modern economic and social systems. It is important to appreciate the historical context and boldness of Williams' arguments, which continue to resonate in discussions about the legacies of slavery and colonialism in modern economies.